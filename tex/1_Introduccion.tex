\capitulo{1}{Introducción}

Green In House es una maceta educativa móvil diseñada para fomentar el aprendizaje de los niños y promover su conexión con la naturaleza. Es un producto que combina la tecnología y la educación sobre el medioambiente de una manera interactiva y práctica.

Green In House se trata de una maceta fabricada con material ecológico (PETG reciclado), el cual se imprime utilizando tecnología de impresión 3D. Green In House no es una simple maceta para plantas, ya que incorpora sensores y una aplicación móvil interactiva para brindar una experiencia de aprendizaje completa a su usuario.

El objetivo de Green In House es mantener un registro temporal de los factores clave que influyen en el crecimiento de las plantas y enseñar a los niños sobre el cuidado de las mismas, concienciándolos sobre la importancia de la naturaleza. A través de esta maceta educativa, los niños pueden aprender a asumir responsabilidades, como regar las plantas y proporcionarles un entorno adecuado.

La aplicación móvil complementaria permite a los usuarios acceder a información detallada sobre las plantas, recibir y generar consejos de cuidado y realizar un seguimiento del crecimiento de sus plantas. Además, la aplicación fomenta la interacción social al permitir a los usuarios compartir sus experiencias y consejos con otros usuarios.

Green In House se enfoca en mantener un control sobre aspectos clave para el crecimiento de las plantas, como la temperatura, la humedad y la cantidad de luz. Los sensores incorporados en la maceta educativa permiten a los usuarios monitorizar y ajustar estos factores para garantizar un entorno óptimo para el desarrollo de las plantas.




% TODO

Estrucutra de la memoria y del resto de materiales entregados.
