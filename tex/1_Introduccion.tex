\capitulo{1}{Introducción}

Green In House es una maceta educativa móvil diseñada para fomentar el aprendizaje de los niños y promover su conexión con la naturaleza. Es un producto que combina la tecnología y la educación ambiental de una manera interactiva y práctica.

Se trata de una maceta fabricada con materiales ecológicos, como el PETG reciclado, que se imprime utilizando tecnología de impresión 3D. Green In House no solo sirve como un macetero para plantas, sino que incorpora sensores y una aplicación móvil interactiva para brindar una experiencia de aprendizaje completa.

El objetivo de Green In House es mantener un registro temporal de los factores clave que influyen en el crecimiento de las plantas y enseñar a los niños sobre el cuidado de las mismas, concienciandolos sobre la importancia de la naturaleza. A través de la maceta educativa, los niños aprenden a asumir responsabilidades, como regar las plantas y proporcionarles un entorno adecuado.

La aplicación móvil complementaria permite a los niños acceder a información detallada sobre las plantas, recibir y generar consejos de cuidado y realizar un seguimiento del crecimiento de sus plantas. Además, la aplicación fomenta la interacción social al permitir a los niños compartir sus experiencias y consejos con otros usuarios.

Green In House también destaca por su enfoque en aspectos clave para el crecimiento de las plantas, como la temperatura, la humedad y la cantidad de luz. Los sensores incorporados en la maceta educativa permiten a los niños monitorizar y ajustar estos factores para garantizar un entorno óptimo para el desarrollo de las plantas.

En resumen, Green In House es una maceta educativa móvil que combina tecnología, educación ambiental y aprendizaje práctico. Proporciona a los niños la oportunidad de cultivar sus propias plantas, aprender sobre el cuidado de la naturaleza y adquirir habilidades valiosas para el cuidado del medio ambiente.




#TODO

Estrucutra de la memoria y del resto de materiales entregados.
