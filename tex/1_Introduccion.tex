\capitulo{1}{Introducción}

Green In House es una maceta educativa móvil diseñada para fomentar el aprendizaje de los niños y promover su conexión con la naturaleza. Es un producto que combina la tecnología y la educación sobre el medioambiente de una manera interactiva y práctica.

Green In House se trata de una maceta fabricada con material ecológico (PETG reciclado), el cual se imprime utilizando tecnología de impresión 3D. Green In House no es una simple maceta para plantas, ya que incorpora sensores y una aplicación móvil interactiva para brindar una experiencia de aprendizaje completa a su usuario.

El objetivo de Green In House es mantener un registro temporal de los factores clave que influyen en el crecimiento de las plantas y enseñar a los niños sobre el cuidado de las mismas, concienciándolos sobre la importancia de la naturaleza. A través de esta maceta educativa, los niños pueden aprender a asumir responsabilidades, como regar las plantas y proporcionar un entorno adecuado.

La aplicación móvil complementaria permite a los usuarios acceder a información detallada sobre los factores que afectan al desarrollo de las plantas, recibir y generar consejos de cuidado, facilitado al usuario poder realizar un seguimiento del crecimiento de sus plantas. 

Green In House se enfoca en mantener un control sobre aspectos clave para el crecimiento de las plantas, como la temperatura, la humedad y la cantidad de luz. Los sensores incorporados en la maceta educativa permiten a los usuarios monitorizar y ajustar estos factores para garantizar un entorno óptimo para el desarrollo de las plantas.

A continuación se describe la estructura de la memoria y del resto de materiales entregados.
\begin{itemize}
  \item \textbf{Introducción:} breve descripción del problema a resolver y de la solución propuesta. Estructura de la memoria y listado de materiales adjuntos.
  \item \textbf{Objetivos del proyecto:} explicación de los objetivos que persigue el proyecto.
  \item \textbf{Conceptos teóricos:} breve explicación de los conceptos teóricos clave para la comprensión del proyecto.
  \item \textbf{Técnicas y herramientas:} listado de técnicas metodológicas y herramientas utilizadas para gestión y desarrollo del proyecto.
  \item \textbf{Aspectos relevantes del desarrollo:} exposición de aspectos relevantes que hubo que valorar antes y durante la realización del proyecto.
  \item \textbf{Trabajos relacionados:} productos competidores actualmente existentes en el mercado
  \item \textbf{Conclusiones y líneas de trabajo futuras:} conclusiones obtenidas tras la realización del proyecto y posibilidades de mejora de la solución aportada.
\end{itemize}

Junto a la memoria se proporcionan los siguientes anexos:

\begin{itemize}
    \item \textbf{Plan del proyecto software:} planificación temporal y estudio de viabilidad del proyecto.
    \item \textbf{Especificación de requisitos del software:} se describe la fase de análisis; los objetivos generales, el catálogo de requisitos del sistema y la especificación de requisitos funcionales y no funcionales.
    \item \textbf{Especificación de diseño:} se describe la fase de diseño desglosada en el diseño de datos, el diseño procedimental y el diseño arquitectónico.
    \item \textbf{Manual del programador:} recoge los aspectos más relevantes relacionados con el código fuente de la aplicación (estructura, compilación, instalación, ejecución, pruebas, etc.).
    \item \textbf{Manual de usuario:} guía de usuario para el correcto manejo de la aplicación.
\end{itemize}

\section{Materiales adjuntos}\label{materiales-adjuntos}

Los materiales que se adjuntan con la memoria son: 
\begin{itemize}
    \item Aplicación para Raspberry Pi con scripts de instalación y configuración del entorno y código fuente de la aplicación.
    \item Aplicación multiplataforma desarrollada en Flutter utilizando Flutter Flow.
\end{itemize}

Todos estos recursos están accesibles públicamente desde los siguientes repositorios:
\begin{itemize}
    \item Repositorio del proyecto para maceta \cite{GreenInHouse:repo:Maceta}.
    \item Repositorio del proyecto para App móvil \cite{GreenInHouse:repo:AppMovil}.
    \item Repositorio del proyecto para documentación \cite{GreenInHouse:repo:Documentación}.
\end{itemize}