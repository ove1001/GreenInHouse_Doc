\apendice{Plan de Proyecto Software}
A continuación se procede a especificar detalladamente como se ha realizado el plan de desarrollo software de Green In House.
\section{Introducción}
La planificación es una etapa fundamental en el desarrollo de cualquier proyecto software. A continuación, se expone por qué es importante realizar una planificación adecuada:
\begin{itemize}
    \item \textbf{Organización y estructura:} La planificación permite establecer una estructura clara y organizada para el proyecto. Define los objetivos, los recursos necesarios, los plazos de entrega y las tareas específicas que se deben realizar para llevar a cabo el proyecto. Esto ayuda a evitar confusiones y asegura que todos los miembros del equipo estén alineados en cuanto a las metas y el enfoque del proyecto.
    \item \textbf{Estimación de recursos y tiempos:} La planificación ayuda a determinar los recursos necesarios para llevar a cabo el proyecto, en términos de personal, de software y de presupuesto. Además, permite estimar los tiempos de ejecución de cada tarea y establecer un cronograma realista. Esto es esencial para gestionar eficientemente los recursos y evitar retrasos y sobre costes.  
    \item \textbf{Identificación de riesgos y mitigación:} Durante la planificación, se pueden identificar posibles riesgos y obstáculos que podrían surgir durante el desarrollo del proyecto. Esto permite tomar medidas preventivas y establecer estrategias para minimizar su impacto en el cronograma y en la calidad del proyecto. Al anticipar y abordar los riesgos de manera pro activa, se reducen las posibilidades de enfrentar problemas imprevistos en un futuro.  
    \item \textbf{Asignación de tareas y responsabilidades:} La planificación ayuda a definir claramente las tareas específicas que deben ser realizadas y asignar responsabilidades a los miembros del equipo. Esto promueve la colaboración y la eficiencia, ya que cada persona conoce cuál es su rol en el proyecto y qué se espera de ella. Además, facilita la coordinación entre los miembros del equipo y permite realizar un seguimiento más preciso del progreso de cada tarea.
    \item \textbf{Control y seguimiento del progreso:} Una vez que el proyecto se ha iniciado, contar con una planificación adecuada facilita el control y seguimiento del progreso. Esto permite comparar el avance real con el planificado, identificar desviaciones y tomar medidas correctivas cuando sea necesario. Esto ayuda a garantizar que se cumplan los objetivos establecidos.
\end{itemize}

\section{Planificación temporal}
Al comenzar con el proyecto, se planteó utilizar una metodología ágil. En este caso se decidió utilizar Scrum. Aunque no se a seguido completamente en todos los aspectos que especifica esta metodología, ya que el equipo de desarrollo únicamente está formado por una persona, la cual hace todos los roles. Debido a eso, no se han realizado reuniones diarias. Las reuniones que se han realizado son las de seguimiento de los sprint, con los tutores del proyecto Raúl Martincorena y Jose Antonio Canepa y con Yeray Pescador. Durante estas reuniones se han definido los objetivos importantes a desarrollar durante el nuevo sprint, se ha hablado de los objetivos conseguidos del sprint anterior y los objetivos que no se han conseguido realizar, se han incluido de nuevo en el siguiente sprint. No se ha empleado ninguna herramienta de seguimiento de los sprint estilo ZenHUB, aunque habría sido una buena idea utilizarla para tener un mejor control del proceso iterativo del proyecto. la duración normal de los \textit{sprints} ha sido de dos semanas, hasta llegar a las fases finales, en las que está duración ha sido modificada a una semana.

\subsection{Sprint 0 (30/1/2023 - 12/2/2023)}

En la reunión de este sprint inicial, lo que se realizó fue hacer una propuesta inicial de lo que sería el TFG, y hacer una lluvia de ideas sobre qué objetivos básicos se debería buscar cumplir, a qué público debería ir destinado, cuál sería la función principal del proyecto, como se haría la comunicación de los datos con sistemas externos y varios aspectos de carácter general

Como objetivos a cumplir en el primer sprint, se propusieron decidir las funcionalidades principales que se iban a implementar e investigar qué tecnologías podría utilizar para resolverlo. 

Las tareas que se realizaron durante este sprint, fue la toma de decisión de que IDE utilizar, que tipo de microcontrolador utilizar (Arduino o Raspberry pi), qué factores de la planta habría que medir, cómo almacenar los datos (en archivos o en base de datos), investigar como realizar la comunicación de los datos con sistemas exteonrs, decidir qué metodología utilizar, realizar un primer enfoque del diseño y empezar con la parte de desarrollo de la parte de qué datos utilizar.

\subsection{Sprint 1 (13/2/2023 - 26/2/2023)}

En la reunión de este \textit{sprint}, lo que se hizo fue debatir entre todos, cuales de los objetivos que había pensado, sería buena idea implementar, definir una serie de funcionalidades básicas con las que empezar a trabajar y decidir  el alcance inicial del proyecto para no abordar más aspectos de los necesarios. Se aceptó la utilización de Raspberry Pi como entorno de desarrollo a utilizar por su posibilidad de cubrir todos los aspecto necesarios del sistema y se decidió utilizar una base de datos como sistema de almacenamiento persistente e implementar toda la aplicación en Python al ser un lenguaje óptimo para desarrollar prototipos y tener ya conocimientos te he dicho lenguaje. Para realizar la comunicación de los datos con sistemas externos se decidió realizar a través de comunicación por red utilizando un servidor API REST 

Como objetivos a cumplir en este \textit{sprint}, se estableció realizar la instalación de el sistema operativo Raspbian en la Raspberry, iniciar con el desarrollo del modelo de datos, comenzar con la implementación de la base de datos del sistema e investigar como realizar la lectura de los sensores.

Las tareas realizadas durante este \textit{sprint} fue la instalación de Raspbian, realizar el diseño inicial de las clases que se iban a utilizar en el sistema, definiendo sus atributos y sus jerarquías, e implementar la base de datos utilizando Python y la librería SQLAlchemy. También se comenzó con el desarrollo de los scripts de instalación del sistema para automatizar y simplificar la instalación y se investigó el uso de la librería AdaFruit para leer los sensores del sistema.

\subsection{Sprint 2 (27/2/2023 - 12/3/2023)}
En la reunión de este \textit{sprint}, lo que se hizo fue analizar el modelo de datos que había diseñado para ver si cubría las necesidades de los objetivos propuestos, resolver dudas que tenía respecto al uso de SQLAlchhemi para resolver las consultas a la base de datos utilizando el ORM y explicar el primer acercamiento con el trabajo con sensores en Raspberry. Se propusieron varias mejoras del modelo de datos y se estudiaron las posibilidades de las consultas mediante el ORM. Durante esta reunión se introdujo un problema que no se había contemplado anteriormente y era decidir cómo el usuario iba a introducir en el sistema las credenciales de su red WiFi.

Como objetivos a cumplir en este \textit{sprint}, se estableció empezar con el desarrollo de la lectura de sensores utilizando aAdaFruit, implementar las mejoras propuestas en el modelo de datos y en el gestor de la base de datos para realizar las consultas. También se propuso como objetivo empezar a investigar cómo desarrollar posteriormente el servidor API Rest.

Las tareas realizadas durante este \textit{sprint} fue la implementación de las mejoras propuestas en el modelo de datos, la mejora de las consultas realizadas por el gestor de la base de datos utilizando el ORM e intentar comunicar con todos los sensores por separado, para ver como había que trabajar con cada sensor.

\subsection{Sprint 3 (13/3/2023 - 26/3/2023)}
En la reunión de este \textit{sprint}, lo que se hizo fue analizar el estado actual del modelo de datos y del gestor de la base de datos, dando por buena la aproximación actual. Se debatió sobre los resultados que había obtenido con las primeras pruebas leyendo los sensores desde Raspberry Pi y los problema que había tenido para ponerlos en funcioamiento. También se debatió sobre qué lirerías utilizar para implementar la API Rest y se decidió utilizar Swagger y Conexxion.

Como objetivos a cumplir en este \textit{sprint}, se estableció definir el funcionamiento de los sensores y crear una interfaz general para todos ellos, la cual implemente cada sensor para sobre escribir el método de lectura en cada clase de manera específica para dicho sensor y poder trabajar con todos los sensores de la misma manera.

Las tareas que se llevaron a cabo en este \textit{sprint} fueron las de crear una interfaz para los sensores electrónicos y una factoría que se encargase de mapear automáticamente los datos almacenados en la base de datos para determinar el objeto de sensor a instanciar en base al modelo declarado y utilizase los datos de las patillas para saber donde estaba conectado el sensor.


\subsection{Sprint 4 (27/3/2023 - 9/4/2023)}
En la reunión de este \textit{sprint} se debatió la estructura que había realizado para trabajar con los sensores electrónicos, y se analizó su eficiencia y escalabilidad si necesitaba meter nuevos sensor durante el proyecto. También se debatió sobre la capa de servicio que estaba implementando para hacer uso del gestor de la base de datos y separar así la lógica del modelo de datos y se comenzó a indagar sobre qué librerías utilizar para implementar la API Rest y se decidió utilizar Swagger y Conexxion.

Como objetivos a cumplir en este \textit{sprint}, se estableció continuar desarrollando la capa de servicios y comenzar con el desarrollo de la API Rest, utilizando Swagger y diseñar un script en Bash para que realizase la lectura periódica de los sensores del sistema, gestionando el arranque de periódico mediante un hilo controlado por el propio sistema operativo.

Las tareas que se realizaron durante este \textit{sprint} fueron las de terminar de crear la capa de servicios que utiliza al gestor de la base de datos y empezar a crear la API REST, encargada de gestionar la comunicación con servicios externos. 

\subsection{SSprint 5 (10/4/2023 - 23/4/2023)}
En la reunión de este \textit{sprint} se debatió los problemas encontrados al trabajar con sensores analógicoa, al no tener Raspberry un conversor analógico digital integrado y se estudió las posibles soluciones. También se habló sobre el uso de conexión en remoto por SSH a la Raspberry ya que era muy ineficiente tener que desarrollar el código en la propia Raspberry.

Como objetivos de este \textit{sprint} se propuso implementar una solución para leer los sensores que faltaban en el sistema, continuar con el desarrollo de la API REST e instalar SSH en la Raspberry para hacer trabajo en remoto sobre ella.

Las tareas que se realizaron durante este \textit{sprint} fueron las de investigar e implementar alternativas para leer los sensores que estaban dando problemas, configurar la comunicación SSH, actualizar los scripts de instalación automática del sistema y continuar con el desarrollo de la API REST y la capa de servicios

\subsection{Sprint 6 (24/4/2023 - 14/5/2023}
En la reunión de este \textit{sprint} se hablo sobre como había resuelto el problema con la lectura de los sensores analógicos, si había podido establecer la comunicación por ssh. En esta reunión se planteó el problema que había tenido al realizar el cambio de red cableada a red wifi en la raspberry pi, ya que no mi Raspberry no tiene WiFi integrado, y la instalación de un USB wifi estaba dando problemas.

Como objetivos del \textit{sprint} se propuso solucionar el problema con la conectividad WiFi, continuar con el desarrollo d ela API REST y estudiar como desarrollar unos \textit{scripts} que me permitiesen lanzar los trabajos de lecturas de sensores miediante hilos del sistema operativo.

Las tareas que se realizaron durante este \textit{sprint} fueron las de conseguir instalar el USB WiFi para poder tener red WiFi en la Raspberr y desarrollar unos scripts que lanzasen durante el arranque del sistema operativo el servicio de lectura de sensores y el servidor API REST.

\subsection{Sprint 7 (15/5/2023 - 28/5/2023)}
En la reunión de este \textit{sprint} se hablo sobre cómo había solucionado el problema con el USB WiFi y sobre el funcionamiento de los scripts de lectura y los problemas que había tenido a la hora de establecer que se lanzasen automáticamente durante el arranque. Se comenzó a debatir cómo realizar la aplicación móvil, como conectar la Raspberry a la red WiFi, y cómo enfocar la memoria.

Como objetivos del \textit{sprint} se propuso comenzar con el desarrollo de la aplicación móvil e intentar comunicar por Bluetooth con la Raspberry para enviarle las credenciales del WiFi. También se decidió comenzar con la memoria.

Las tareas que se realizaron durante este \textit{sprint} fueron comenzar con el desarrollo de la aplicación móvil e intentar comunicar por Bluetooth con la Raspberry y hacer uso del servidor API REST cuando tuviera conexión de red. también se comenzó con el desarrollo de la memoria.


\subsection{Sprint 8 (29/5/2023 - 4/6/2023)}
En la reunión de este \textit{sprint} se hablo sobre los problemas encontrados al hacer uso del Bluetooth desde Flutter Flow, y qué alternativa utilizar al no poder enviar las credenciales del WiFi por Bluetooth. se habló sobre el uso de los \textit{endpoint} de la API Rest desde Flutter Flow y se revisó los primeros pasos de la memoria.

Como objetivos del \textit{sprint} se propuso desarrollar una alternativa para introducir las credenciales WiFi en la Raspberry y continuar con el desarrollo de la aplicación móvil y de la memoria. También se decidió crear un sistema de consejos a seguir para el cuidado de la planta.

Las tareas que se realizaron durante este \textit{sprint} fueron realizar una aplicación en una pantalla táctil conectada a la Raspberry que permitiera introducir las credenciales WiFi, crear el sistema de consejos de la planta, añadir más servicios de API REST y continuar con el desarrollo de la aplicación móvil y de la memoria.


S\subsection{Sprint 9 (5/6/2023 - 11/6/2023)}
En la reunión de este \textit{sprint} se hablo sobre el estado de la API REST, la aplicación de la pantalla táctil para introducir las credenciales de la red WiFi, cómo graficar los datos de los registros en la aplicación móvil y los problemas que me estaba dando, cómo se había resuelto la parte de los consejos, y se continuó revisando la memoria

Como objetivos del \textit{sprint} se propuso terminar la aplicación móvil, mostrando únicamente las gráficas de los sensores en intervalos temporales y continuar con el desarrollo de la memoria y los anexos.

Las tareas que se realizaron durante este \textit{sprint} fueron terminar de desarrollar la aplicación móvil para que pudieran mostrar las gráficas de los sensores agrupados por intervalos temporales y continuar con la documentación de la memoria y los anexos.


\subsection{Sprint 10 (11/6/2023 - 18/6/2023)}
En la reunión de este \textit{sprint} se hablo sobre el estado general de la aplicación de la maceta y del móvil, dando por terminadas estas partes. Se decidió centrar los esfuerzos en terminar la memoria y los anexos.

Como objetivos del \textit{sprint} se propuso avanzar lo máximo posible en la documentación de la memoria y los anexos.

Las tareas que se realizaron durante este \textit{sprint} fueron para desarrollar lo máximo posible la parte de la memoria y los anexos.


\subsection{Sprint 11 (19/6/2023 - 25/6/2023)}
En la reunión de este \textit{sprint} se hablo sobre el estado actual de la memoria, se hizo una revisión de los puntos realizados y se planteó cómo enfocar los puntos faltantes.

Como objetivos del \textit{sprint} se propuso terminar de desarrollar la memoria y avanzar lo máximo posible en la parte de anexos

Las tareas que se realizaron durante este \textit{sprint} fueron para terminar la parte de la memoria y avanzar lo máximo posible en la parte de anexos.


\subsection{Sprint 12 (26/6/2023 - 5/7/2023)}
En la reunión de este ultimo \textit{sprint} se  revisó completamente de la memoria y los anexos, a falta de algunos apartados de anexos.

Como objetivos del \textit{sprint} se propuso terminar la parte de anexos y revisar que el sistema se pudiera instalar correctamente de manera automática en un entorno limpio, utilizando los \textit{scripts} de instalación.

Las tareas que se realizaron durante este \textit{sprint} fueron terminar la parte de anexos y verificar que el sistema se pudiera instalar correctamente de manera automática en un entorno limpio, utilizando los \textit{scripts} de instalación.


\section{Estudio de viabilidad}
Para poder determinar si fabricar a gran escala Green In House podría ser rentable, o no, es necesario realizar correctamente un estudio de viabilidad.
    \subsection{Viabilidad económica}
    El estudio de viabilidad económica es una etapa fundamental para determinar la rentabilidad y viabilidad financiera del proyecto Green In House. Este estudio permitirá evaluar los costos y beneficios asociados al desarrollo y funcionamiento de Green In House.
    Para llevar a cabo el análisis económico, es necesario considerar diferentes aspectos:
        \subsubsection{Costes de desarrollo inicial}
        Se deben estimar los costes asociados al desarrollo del software, la adquisición del hardware (ordenador, dispositivo móvil, impresoras 3D, Raspberry Pi, sensores y componentes electrónicos), así como los gastos de diseño y construcción de la maceta. El tiempo para esta estimación será de 6 meses.
            \subsubsubsection{Costes de personal}
            El proyecto lo realizará un desarrollador, a tiempo parcial (20 horas semanales) durante los 6 meses especificados, teniendo en cuenta el siguiente salario estimado. Estos gastos de personal se reflejan en la tabla \ref{tab:costes de personal de desarrollo inicial}.
            \begin{table}[H]
                \centering
                \caption{Costes de personal}
                \begin{tabular}{|l|c|}
                    \hline
                    Concepto & Coste \\
                    \hline
                    Salario mensual neto & 606€ \\
                    Retención IRPF (15\%) & 165€ \\
                    Seguridad Social (29.9\%) & 329€ \\
                    \hline
                    Salario mensual bruto & 1100€ \\
                    \hline
                    Total 6 meses & 6600€ \\
                    \hline
                \end{tabular}
                \label{tab:costes de personal de desarrollo inicial}
            \end{table}
            \subsubsubsection{Costes de hardware}
            En este apartado se reflejan los costes que han sido necesarios, para desarrollar el dispositivo hardware que hemos necesitado para este proyecto. Estos gastos de hardware se reflejan en la tabla \ref{tab:costes de hardware de desarrollo inicial}.
            \begin{table}[H]
                \centering
                \caption{Costes de hardware}
                \begin{tabular}{|l|c|c|c|}
                    \hline
                    Concepto & Coste & Cantidad & Coste total \\
                    \hline
                    Dispositivo móvil & 400€ & 1 & 400€ \\
                    Ordenador & 1500€ & 1 & 1500€ \\
                    Impresora 3D & 1000€ & 1 & 1000€ \\
                    \hline
                    Total & & & 2900€ \\
                    \hline
                \end{tabular}
                \label{tab:costes de hardware de desarrollo inicial}
            \end{table}
            \subsubsubsection{Costes de software}
            En este apartado se definen los costes de las licencias de software que no son gratuitos. Estos gastos de software se reflejan en la tabla A.3.
            \begin{table}[H]
                \centering
                \caption{Costes de software}
                \begin{tabular}{|l|c|c|}
                    \hline
                    Concepto & Coste & Coste total \\
                    \hline
                    Windows 10 Pro & 200€ & 200€ \\
                    Flutter flow (suscripción) & 30€ al mes & 180€ \\
                    \hline
                    Total &  & 380€ \\
                    \hline
                \end{tabular}
                \label{tab:costes de software de desarrollo inicial}
            \end{table}
            \subsubsubsection{Costes de producción de prototipo}
            En este apartado se definen los costes de materiales utilizados para realizar la producción del prototipo, sumando la amortización de gastos. Estos gastos de producción de la maceta se reflejan en la tabla \ref{tab:costes de producción de prototipo de desarrollo inicial}.
            \begin{table}[H]
                \centering
                \caption{Costes de producción por maceta}
                \begin{tabular}{|l|c|c|c|}
                    \hline
                    Concepto & Coste & Cantidad & Coste total \\
                    \hline
                    Raspberry Pi & 60€ & 1 & 60€ \\
                    Tarjeta SD & 7€ & 1 & 7€ \\
                    Batería & 4€ & 1 & 4€ \\
                    Pantalla táctil & 12€ & 1 & 12€ \\
                    Cable HDMI & 1,5€ & 1 & 1,5€ \\
                    Cable USB & 0,5€ & 1 & 0,5€ \\
                    Sensor FC28 & 3€ & 1 & 3€ \\
                    Sensor DHT11 & 3€ & 1 & 3€ \\
                    Sensor BH1750 & 2€ & 2 & 4€ \\
                    Cableado & 0,3€ metro & 5 & 1,5€ \\
                    Plástico maceta & 1€ kilo & 2 & 2€ \\
                    Impresión de maceta & 0,3€ KWh & 5 & 1,5€ \\
                    \hline
                    Total & & & 100,00€ \\
                    \end{tabular}
                    \label{tab:costes de producción de prototipo de desarrollo inicial}
                \end{table}
            \subsubsubsection{Costes de documentación}
            En este apartado se definen los costes de los materiales de documentación. Estos gastos de documentación se reflejan en la tabla \ref{tab:costes de documentación de desarrollo inicial}.
            \begin{table}[H]
                \centering
                \caption{Costes de producción por maceta}
                \begin{tabular}{|l|c|c|c|}
                    \hline
                    Concepto & Coste & Cantidad & Coste total \\
                    \hline
                    Impresión de memoria & 20€ & 1 & 20€ \\
                    \hline
                    Total & & & 20,00€ \\
                    \end{tabular}
                    \label{tab:costes de documentación de desarrollo inicial}
                \end{table}
            
            \subsubsubsection{Costes totales}
            En este apartado se realiza el sumatorio total de todos los gastos estimandos asociados al desarrollo del Green In House. Este resumen de gastos totales se reflejan en la tabla \ref{tab:costes totales de desarrollo inicial}.
            \begin{table}[H]
                \centering
                \caption{Costes totales}
                \begin{tabular}{|l|c|}
                    \hline
                    Concepto & Coste \\
                    \hline
                    Personal & 6.600€ \\
                    Hardware & 2.900€ \\
                    Software & 380€ \\
                    Fabricación prototipo & 100€ \\
                    Documentación & 20€ \\
                    \hline
                    Total & 10.000€ \\
                    \hline
                \end{tabular}
                \label{tab:costes totales de desarrollo inicial}
            \end{table}
            Para desarrollar el proyecto el único gasto real que he tenido que afrontar son los gastos de materiales para la fabricación del prototipo, ya que he utilizado mi ordenador actual con Windows, las licencias de software han sido cedidas por el fabricante al ser un estudiante universitario, la impresora 3D ha sido facilitado su uso sin coste más allá del material gastado y no se ha tenido que pagar a ningún desarrollador, ya que esa tarea la he realizado yo.

        \subsubsection{Análisis de mercado}
        Para conocer la potencial demanda de Green In House hay que realizar un estudio de mercado. Esto implica analizar el tamaño del mercado, identificando a los potenciales clientes y evaluando la competencia existente.
            \subsubsubsection{Potenciales clientes}
            Green In House principalmente está diseñado y pensado para niños, pero también puede ser útil para los adultos y hay varias segmentaciones de clientes que podrían encontrar este producto interesante:
            \begin{itemize}      
                \item \textbf{Jardineros aficionados}: Los adultos que disfrutan del jardín y quieren aprender más sobre el cuidado de las plantas podrían beneficiarse de las características interactivas y educativas de Green In House.
                \item \textbf{Amantes de la tecnología}: Aquellos que están interesados en la tecnología, especialmente la tecnología que se integra con la vida cotidiana de formas nuevas e interesantes, pueden ver el valor en un producto como Green In House.
                \item \textbf{Profesores y educadores}: Los profesores podrían utilizar Green In House como una herramienta educativa en las aulas para enseñar sobre ciencias naturales, ecología y biología, o para fomentar la responsabilidad y el cuidado del medio ambiente.
                \item \textbf{Padres}: Los padres que quieren proporcionar experiencias educativas prácticas para sus hijos en casa podrían estar interesados en Green In House.
                \item \textbf{Personas que viven en espacios reducidos}: Para aquellos que viven en apartamentos o casas sin jardín, Green In House ofrece una forma de tener y cuidar plantas en espacios interiores.
                \item \textbf{Empresas de bienestar en el trabajo}: Las empresas que buscan mejorar el bienestar de sus empleados pueden estar interesadas en utilizar Green In House para animar a los empleados a tomar descansos activos, cuidar una planta y desconectar de su trabajo por un tiempo.
                \item \textbf{Asistentes de terapia ocupacional y psicólogos}: Green In House podría ser utilizado como una herramienta en la terapia ocupacional o en el tratamiento de trastornos de atención, enseñando a los pacientes a concentrarse en tareas, a ser conscientes del momento presente y a cuidar de algo más.
            \end{itemize}
            \subsubsubsection{Competidores}
            Actualmente en el mercado ya se ofrecen diversas alternativas a este producto, aunque todas tienen un enfoque más autónomo puramente dicho, lo cual en Green In House se ha eliminado esa parte de autosuficiencia, para involucrar lo máximo posible al usuario  en el desarrollo de su planta (en este caso el niño):
            \begin{itemize}      
                \item \textbf{Click and Grow :} es un jardín interior inteligente que permite cultivar plantas en casa. Viene con cápsulas de semillas que se insertan en el jardín inteligente para un crecimiento sin problemas. Los sensores y el sistema automatizado de riego aseguran que las plantas reciban la cantidad adecuada de agua, luz y nutrientes. Los precios varían desde alrededor de 100€ hasta 200€ dependiendo del tamaño del jardín inteligente.
                \item \textbf{AeroGarden :} es otro sistema de jardín interior inteligente que permite cultivar plantas en interiores durante todo el año. También utiliza semillas en cápsulas y tiene un sistema de iluminación LED ajustable y un panel de control para recordar cuándo agregar agua y nutrientes. Los precios oscilan entre 100€ y 300€ dependiendo del modelo.
                \item \textbf{Planty Square :} es un jardín modular inteligente que permite a los usuarios cultivar varias plantas a la vez. Cada módulo puede cultivar una planta y los módulos se pueden conectar entre sí para formar un jardín más grande. Cuenta con una aplicación que envía notificaciones para regar las plantas. El precio ronda los 100€ .
                \item \textbf{SproutsIO :} otro sistema es un jardín interior de alta tecnología que permite cultivar plantas sin tierra. Utiliza la hidroponía y la aeroponía para cultivar plantas y tiene una aplicación que permite a los usuarios monitorear y controlar su jardín desde su teléfono. El precio es más elevado, alrededor de 800€.
            \end{itemize}
            Es importante aclarar que estos precios pueden variar por región debido a factores como los impuestos, los costos de envío y la tasa de cambio. Todos estos productos tienen en común la idea de utilizar la tecnología para facilitar el cultivo de plantas en interiores, pero cada uno tiene sus propias características únicas, al igual que Green In House. En la tabla \ref{tab:comparativa competidores} se muestra un análisis comparativo de las funcionalidades que aporta Green In House frente a sus competidores actualmente existentes en el mercado.
            \begin{table}[ht]
                \resizebox{13cm}{!} {
                \begin{tabular}{|l|l|l|l|l|l|}
                \hline   & \textbf{Green In House} & \textbf{Click and Grow} & \textbf{AeroGarden} & \textbf{Planty Square} & \textbf{SproutsIO} \\ \hline
                \textbf{Precio}   & Medio   & Medio/Alto   & Alto   & Medio/Alto   & Muy alto   \\ \hline
                \textbf{Conectividad}   & Sí   & Sí   & Sí   & Sí   & Sí   \\ \hline
                \textbf{Completamente automatizado}     & No   & Sí   & Sí   & Sí   & Sí   \\ \hline
                \textbf{Educación Ambiental}   & Sí   & No   & No   & No   & No   \\ \hline
                \textbf{Personalización}   & Sí   & No   & No   & Parcial   & No   \\ \hline
                \textbf{Posibilidad de expansión}   & Sí    & No    & No    & No    & No    \\ \hline
                \textbf{Open Source}   & Sí    & No    & No    & No    & No    \\ \hline
                \end{tabular}
                }
                \caption{Comparativa de Green In House con la competencia}
                \label{tab:comparativa competidores}
            \end{table}
        
        \subsubsection{Plan de negocio}
        Para poder definir un precio de venta al público he realizado un plan de negocio básico, sin entrar en demasiado detalle, asumiendo que el dinero de la inversión inicial se posee y no sería necesario pedir créditos. Debido a ello no se han tenido en cuenta intereses aplicados en el tiempo. Para realizar este plan he estimado los costes asociados al sueldo del desarrollador, la adquisición del hardware (ordenador, dispositivo móvil, impresoras 3D, Raspberry Pi, sensores y componentes electrónicos), así como los gastos de diseño y construcción de las maceta. Además, se deben incluir los costos de personal (un desarrollador), los costes de alquiler de oficina, los costes de servicios como luz, gas e internet y ver como amortizar dichos gastos en el precio de las macetas educativas vendidas. Para calcular el sobrecoste derivado de la amortización de costes de fabricación que habría que aplicar al precio de los materiales de cada maceta, he estimado como tiempo de amortización de gastos iniciales total de 4 años y una producción y venta mensual de 100 macetas.
            \subsubsubsection{Costes de personal}
            El proyecto lo realizará un desarrollador, a tiempo completo, teniendo en cuenta el siguiente salario estimado. Estos gastos de personal se reflejan en la tabla \ref{tab:costes de personal de plan de negocio}.
            \begin{table}[H]
                \centering
                \caption{Costes de personal}
                \begin{tabular}{|l|c|c|}
                    \hline
                    Concepto & Coste & Coste amortizado\\
                    \hline
                    Salario mensual neto & 1212€ & 1,01€ \\
                    Retención IRPF (15\%) & 330€ & 0,28€ \\
                    Seguridad Social (29.9\%) & 658€ & 0,55€\\
                    \hline
                    Salario mensual bruto & 2200€ & 1,83€\\
                    \hline
                    Total 14 pagas & 30800€ & 25,67\\
                    \hline
                \end{tabular}
                \label{tab:costes de personal de plan de negocio}
            \end{table}
            En el apartado de seguridad social se han sumado los costes en contingencias comunes ,desempleo, garantía salarial y la formación profesional.
            \subsubsubsection{Costes de hardware}
            En este apartado pondremos los costes que han sido necesarios, para desarrollar el dispositivo hardware que hemos necesitado para este proyecto. Estos gastos de hardware se reflejan en la tabla \ref{tab:costes de hardware de plan de negocio}.
            \begin{table}[H]
                \centering
                \caption{Costes de hardware}
                \begin{tabular}{|l|c|c|c|c|}
                    \hline
                    Concepto & Coste & Cantidad & Coste total & Coste amortizado \\
                    \hline
                    Dispositivo móvil & 400€ & 1 & 400€ & 0,08€ \\
                    Ordenador & 1500€ & 1 & 1500€ & 0,31€ \\
                    Impresora 3D & 1000€ & 5 & 5000€ & 1,04€ \\
                    \hline
                    Total & & & 6900€ & 1,44€ \\
                    \hline
                \end{tabular}
                \label{tab:costes de hardware de plan de negocio}
            \end{table}
            \subsubsubsection{Costes de software}
            En este apartado se definen los costes de las licencias de software que no son gratuitos. Estos gastos de software se reflejan en la tabla \ref{tab:costes de software de plan de negocio}.
            \begin{table}[H]
                \centering
                \caption{Costes de software}
                \begin{tabular}{|l|c|c|c|}
                    \hline
                    Concepto & Coste & Coste amortizado anual & Coste amortizado \\
                    \hline
                    Windows 10 Pro & 200€ & 50€ & 0,04€ \\
                    Flutter flow (suscripción) & 30€ al mes & 360€ & 0,30€ \\
                    \hline
                    Total & 230€ & 410€ & 0,34€ \\
                    \hline
                \end{tabular}
                \label{tab:costes de software de plan de negocio}
            \end{table}
            \subsubsubsection{Costes de servicios}
            En este apartado se definen el resto de gastos del proyecto. Estos gastos de servicios se reflejan en la tabla \ref{tab:costes de servicios de plan de negocio}.
            \begin{table}[H]
                \centering
                \caption{Costes varios}
                \begin{tabular}{|l|c|c|c|c|}
                    \hline
                    Concepto & Coste & Cantidad & Coste total & Coste amortizado \\
                    \hline
                    Alquiler de oficina & 600€ & 12 & 7.200€ & 6,00€ \\
                    Internet & 40€ & 12 & 480€ & 0,40€ \\
                    Electricidad & 300€ & 12 & 3600€ & 3,00€ \\
                    Gas & 100€ & 12 & 1200€ & 1,00€ \\
                    \hline
                    Total 1.040€ & & & 12.480€ & 10,40\\
                    \hline
                \end{tabular}
                \label{tab:costes de servicios de plan de negocio}
            \end{table}
            \subsubsubsection{Costes de producción por maceta}
            En este apartado se definen los costes de producción de una maceta. Primeramente se calculan los costes referentes a los gastos de materiales como se reflejan en la tabla \ref{tab:costes de materiales por maceta de plan de negocio}.
            \begin{table}[H]
                \centering
                \caption{Costes de materiales por maceta}
                \begin{tabular}{|l|c|c|c|}
                    \hline
                    Concepto & Coste & Cantidad & Coste total \\
                    \hline
                    Raspberry Pi & 60€ & 1 & 60€ \\
                    Tarjeta SD & 7€ & 1 & 7€ \\
                    Batería & 4€ & 1 & 4€ \\
                    Pantalla táctil & 12€ & 1 & 12€ \\
                    Cable HDMI & 1,5€ & 1 & 1,5€ \\
                    Cable USB & 0,5€ & 1 & 0,5€ \\
                    Sensor FC28 & 3€ & 1 & 3€ \\
                    Sensor DHT11 & 3€ & 1 & 3€ \\
                    Sensor BH1750 & 2€ & 2 & 4€ \\
                    Cableado & 0,3€ metro & 5 & 1,5€ \\
                    Plástico maceta & 1€ kilo & 2 & 2€ \\
                    Impresión maceta & 0,3€ KWh & 5 & 1,5€ \\
                    \hline
                    Total & & & 100,00€ \\
                    \hline
                \end{tabular}
                \label{tab:costes de materiales por maceta de plan de negocio}
            \end{table}
            A estos costes hay que sumarle las amortizaciones de gatos estimados, para calcular el precio de producción total asociado a cada maceta, como se refleja en la tabla \ref{tab:costes de producción por maceta de plan de negocio} .
            \begin{table}[H]
                \centering
                \caption{Costes de producción por maceta}
                \begin{tabular}{|l|c|c|c|}
                    \hline
                    Concepto & Coste & Cantidad & Coste total \\
                    \hline
                    Costes de materiales & 100,00€ & 1 & 100,00€ \\
                    Amortización de personal & 25,67€ & 1 & 25,67€ \\
                    Amortización de hardware & 1,44€ & 1 & 1,44€ \\
                    Amortización de software & 0,34€ & 1 & 0,34€ \\
                    Amortización de servicios & 10,40€ & 1 & 10,40€ \\
                    \hline
                    Total & & & 137,85€ \\
                    \hline
                \end{tabular}
                \label{tab:costes de producción por maceta de plan de negocio}
            \end{table}
            \subsubsubsection{Costes totales}
            En este apartado se realiza el sumatorio total de todos los gastos anuales estimados. Este resumen de gastos totales se reflejan en la tabla \ref{tab:costes totales por maceta de plan de negocio}.
            \begin{table}[H]
                \centering
                \caption{Costes totales}
                \begin{tabular}{|l|r|}
                    \hline
                    Concepto & Coste \\
                    \hline
                    Personal & 30.800€ \\
                    Hardware & 6.900€ \\
                    Software & 410€ \\
                    Varios & 12.480€ \\
                    Producción 1200 macetas & 120.000€ \\
                    \hline
                    Total anual & 170.590€ \\
                    \hline
                \end{tabular}
                \label{tab:costes totales por maceta de plan de negocio}
            \end{table}
            \subsubsubsection{Conclusión de precio de la maceta educativa}
            Amortizando los costes de personal, de hardware, de software y de varios, en cada una de las macetas, sale que el precio para no perder dinero vendiendo las 1200 macetas estimadas al año, tendría que ser 137,85€. El objetivo de producir estas macetas, es obtener más beneficios a parte del salario que se va a cobrar como desarrollador para poder reinvertirlo en mejorar la empresa, por lo que habría que venderlas a un mayor coste. En estos costes, no están incluidos los gastos de envío, ya que dependerán de la zona a la que halla que realizar el envío y se facturarán aparte. El precio estimado de Green In House sería de 180€. 
            Gracias a su versatilidad de software se podrían construir modelos que contengan varias plantas en un mismo macetero con sensores independientes para cada una, sin elevar demasiado los costes, ya que los mayores costes vienen del precio de la Raspberry Pi, la cual puede leer muchísimos más sensores de los que se están utilizando. Además, el precio actual de la Raspberry Pi está muy alto debido a los actuales problemas con la distribución de semiconductores, ya que su precio original era de 30€.
            \subsubsubsection{Beneficios}
            El beneficio estimado de la venta de Green in House vendiéndolo a 180€ y descontando los 137,85€ que costaría fabricar la maceta, da un beneficio de 42,15€ por unidad. Esto multiplicado por las 1200 unidades que se estiman vender, daría unos beneficios anuales de 50.585€.

    
    \subsection{Viabilidad legal}
    Para poder distribuir Green In House es necesario tener en cuenta que hay que cumplir con ciertos requisitos legales, entre ellos el más importante tiene que ver con el tema de la licencia con la que se suministra. Esta licencia no puede ser impuesta arbitrariamente, ya que al utilizar otras librerías es necesario adapte a la licencia con la que cuentan dichas librerías, teniendo que utilizar como mínimo la licencia más restrictiva de todas. Para poder realizar correctamente este análisis a continuación se expone una tabla \ref{tab:Dependencias y versiones} con las dependencias que utiliza Green In House, sus versiones, su uso y su licencia de uso asociada. 

    \begin{longtable}{ p{0.2\columnwidth} p{0.1\columnwidth} p{0.4\columnwidth} p{0.3\columnwidth} }
	\caption{Dependencias y versiones}\\
        \label{tab:Dependencias y versiones}\\
	\hline
	\textbf{Dependencia} & \textbf{Versión} & \textbf{Descripción} & \textbf{Licencia}\\
	\hline
	\endfirsthead
	\textbf{Dependencia} & \textbf{Versión} & \textbf{Descripción} & \textbf{Licencia}\\
	\hline
	\endhead
	Python & 3.9 & Lenguaje de programación general & PSFL \\
	Flutter & 3.7.12 & Lenguaje de programación multiplataforma & BSD3-Clause \\
	Venv & 3.9 & Entorno virtual Python & PSFL \\
	SQLAlchemy & 2.0.0b3 & Librería Python para ORM de base de datos & MIT \\
	SQLite3 & 3.34.1 & Librería C de motor de base de datos SQL ligero & BSD \\
	OpenAPI (swagger) & 2.14.2 & Libería Python para desarrllo de servidor API REST & MIT \\
	Connexion & 2.14.2 & Libería Python para desarrllo de servidor API REST & MIT \\
	Tkinter & 0.1.0 & Libería Python para desarrllo de aplicaciones gráficas & PSFL \\
	AdaFruit Circuit Python DHT & 4.0.2 & Librería Python para trabajo con sensores DHT & MIT \\
	AdaFruit Circuit Python BH1750 & 4.0.2 & Librería Python para trabajo con sensores BH1750 & MIT \\
	AdaFruit Circuit Python MCP3XXX & 4.0.2 & Librería Python para trabajo con ADC MCP3XXX & MIT \\
	Flask & 2.2.5 & Framework de microservicios Python & BSD3-Clause \\
	PyYAML & 6.0 & Librería Python para archivos yaml & MIT \\
	appdirs & 1.4.4 & Librería Python para gestión de directorios & MIT \\
	\hline
    \end{longtable}

    Al revisar la tabla de dependencias utilizas en Green In House, se aprecia que las licencias más comunes son MIT, PSFL, y BSD, las cuales son todas licencias permisivas. Estas licencias permiten la redistribución y modificación del código, incluso en software cerrado o comercial, siempre que se incluya una copia de la licencia original y los derechos de autor. 
    
    Por lo tanto, creo que la elección más acertada de licencia a utilizar para Green In House sería una licencia MIT. La razón por la que he escogido una licencia MIT es porque es simple, fácil de entender, y compatible con prácticamente todas las demás licencias. Además, permite la utilización en aplicaciones comerciales, lo cual abre la posibilidad de monetizar Green In House en el futuro si así lo decidiese. Además, aunque esta licencia permite el uso comercial de Green In House, no ofrecen ninguna garantía y no se responsabilizan de los posibles daños derivados de su uso.
    

    
    