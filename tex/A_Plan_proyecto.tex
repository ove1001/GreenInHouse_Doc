\apendice{Plan de Proyecto Software}

\section{Introducción}



\section{Planificación temporal}

\section{Estudio de viabilidad}

\subsection{Viabilidad económica}

\subsection{Viabilidad legal}

\subsection{potenciales clientes}

Green In House principalmente está diseñado y pensado para niños, pero también puede ser útil para los adultos y hay varias segmentaciones de clientes que podrían encontrar este producto interesante:

\textbf{Jardineros aficionados}: Los adultos que disfrutan del jardín y quieren aprender más sobre el cuidado de las plantas podrían beneficiarse de las características interactivas y educativas de Green In House.

\textbf{Amantes de la tecnología}: Aquellos que están interesados en la tecnología, especialmente la tecnología que se integra con la vida cotidiana de formas nuevas e interesantes, pueden ver el valor en un producto como Green In House.

\textbf{Profesores y educadores}: Los profesores podrían utilizar Green In House como una herramienta educativa en las aulas para enseñar sobre ciencias naturales, ecología y biología, o para fomentar la responsabilidad y el cuidado del medio ambiente.

\textbf{Padres}: Los padres que quieren proporcionar experiencias educativas prácticas para sus hijos en casa podrían estar interesados en Green In House.

\textbf{Personas que viven en espacios reducidos}: Para aquellos que viven en apartamentos o casas sin jardín, Green In House ofrece una forma de tener y cuidar plantas en espacios interiores.

\textbf{Empresas de bienestar en el trabajo}: Las empresas que buscan mejorar el bienestar de sus empleados pueden estar interesadas en utilizar Green In House para animar a los empleados a tomar descansos activos, cuidar una planta y desconectar de su trabajo por un tiempo.

\textbf{Asistentes de terapia ocupacional y psicólogos}: Green In House podría ser utilizado como una herramienta en la terapia ocupacional o en el tratamiento de trastornos de atención, enseñando a los pacientes a concentrarse en tareas, a ser conscientes del momento presente y a cuidar de algo más.