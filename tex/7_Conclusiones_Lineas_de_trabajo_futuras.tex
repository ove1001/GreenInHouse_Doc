\capitulo{7}{Conclusiones y Líneas de trabajo futuras}

A continuación se exponen las conclusiones obtenidas del desarrollo de Green In House y las posibles lineas de trabajo futuras para continuar mejorando Green In House.

\section{Conclusiones}
Actualmente Green In House se encuentra en una fase de desarrollo muy avanzada y se puede considerar un producto mínimo viable. Los objetivos iniciales del proyecto han sido cubiertos completamente, desarrollando un sistema capaz de recoger datos de diversos sensores electrónicos reales, almacenarlos en una base de datos, procesarlos y enviarlos como consultas de una API REST y generar nuevas instancias de elementos del sistema en la base de datos desde peticiones de la API REST. Además se ha diseñado con una estructura modular fácil de ampliar, con capacidad de generar nuevos sensores y plantas dinámicamente a través de servicios de la API REST. También cuenta con una aplicación gráfica que permite al usuario introducir las credenciales su red WiFi doméstica para que Green In House pueda comunicarse a través de red con la aplicación móvil desarrollada para visualizar gráficos de los registros del sistema en diferente periodos de tiempo.

La realización de Green In House me ha permitido investigar diferentes tecnologías y metodologías actuales a un nivel de profundidad considerable, suficiente como para hacerme a la idea de las grandes posibilidades de diseño y desarrollo que otorgan. 
Aunque estoy especializado en la programación de sistemas de automatización industrial mediante PLCs y redes de comunicación industrial para envío de datos entre dispositivos, este ha sido mi primer proyecto de automatización realizado  en la plataforma Raspberry Pi. Esta primera aproximación la he realizado bastante en profundidad, teniendo que investigar cómo hacer uso de su interfaz de comunicación con sensores, sus lenguajes y protocolos soportados, cuales son las tensiones de trabajo que soporta y cómo realizar la conexión con los distintos dispositivos electrónicos que puede manejar.

Esta investigación me ha ayudado a descubrir que Raspberry Pi aporta un entorno de desarrollo ágil muy completo, al soportar Python para el desarrollo de sus aplicaciones. Además cuenta con numerosas librerías específicas de manejo de sensores aparte de las múltiples librerías soportadas por Python de carácter general para realizar tareas como despliegue de un gestor de base de datos ORM y servicios de comunicación HTML como API REST. Además, al ser un sistema con múltiples núcleos, permite la ejecución paralela de tareas mediante hilos, lo cuál es muy útil para poder realizar varias tareas simultáneamente. Raspberry Pi ha sido una gran elección para el desarrollo de Green In House, al tener todas las funcionalidades que necesitaba reunidas en una misma placa. Sin embargo, he podido descubrir, que no es el sistema más adecuado para realizar tareas críticas de lectura de sensores en tiempo real. Esto no ha afectado al proyecto ya que la lectura de los sensores de Green In House se realiza en un intervalo de minutos, pero si se realizase cada poco milisegundos, sería más óptimo utilizar Arduino para leer dichos sensores y después enviar esa información almacenada en paquetes a la Raspberry Pi.

La realización de este proyecto también me ha permitido investigar más en profundidad en el campo de las bases de datos, y los gestores ORM. Durante el grado había desarrollado alguna pequeña base de datos para algunos trabajos, pero siempre siguiendo algún esquema ya dado. Esta es la primera vez que he tenido que discernir por mi mismo cual es la mejor opción de diseño de datos para poder llevar a cabo el proyecto, cuál es la mejor forma de segmentar dichos datos en tablas individuales y cómo interrelacionarlas de manera óptima para tener todos los datos vinculados entre sí.

\section{Líneas de trabajo futuras}
La parte de \textit{backend} está perfectamente diseñada y es completamente funcional en el estado actual, tanto para generar nuevas plantas, tipos de plantas, sensores, consejos, asociar plantas y sensores, dar de baja cualquier elemento del sistema, modificar los datos de elementos del sistema, etc., pero se pueden realizar ciertas mejoras, sobre todo en la parte del \textit{frontend}:
\begin{itemize}
    \item Diseñar la parte de frontend del tal manera que pueda aprovechar todas las funciones que ofrece el \textit{backend}, de una manera intuitiva y llamativa para un niño. Por ello esta parte sería mejor desarrollarla en combinación con alguna persona que tenga un amplio conocimiento en el ámbito del trabajo con niños. 
    \item Mejorar la aplicación de la pantalla táctil incorporada en la maceta para permitir visualizar en tiempo real los datos de los sensores.
    \item Incluir comprobación de contraseña de red WiFi introducida por el usuario para verificar que cumple el estándar WPA, ya que si no lo cumple, almacenar esta credencial en el sistema puede generar un problema permanente en la conectividad de la red WiFi.
    \item En la parte de backend se podría desarrollar la comunicación con nuevos modelos de sensores para ampliar la cantidad de factores físicos que es capaz de medir.
    \item Incluir seguridad en el servidor API REST desarrollando un sistema de verificación de \textit{API KEY} o de \textit{token}.
    \item Generar un sistema de notificaciones que avisase al usuario automáticamente cuando los valores de los sensores se saliesen de los rangos establecidos en los consejos de la planta.
    \item Intentar resolver en Flutter la dirección IP en la que está alojado el servidor API REST a través del hostname GreenInHouse, para poder habilitar el direccionamiento por DHCP y evitar posibles conflictos de red.
    \item Al poder generarse nuevos consejos para las plantas y modificar los existentes, también sería interesante crear una comunidad de usuarios donde pudieran compartir sus experiencias y consejos, para intentar aprender de otras personas. Esto sería muy interesante, ya que va acorde con los objetivos sociales de Green In House y promueve una interacción activa con otras personas.
    \item Generar un plan de pruebas automatizado de pruebas unitarias y de pruebas de integración para poder validar más eficientemente que al añadir un nuevo módulo o funcionalidad, no se ve repercutido el correcto funcionamiento del código existente.
    \item Otra posible mejorar que se podría realizar sería dockerizar la aplicación de Green In House para hacer aún más sencillo su despliegue en nuevos entornos.
    \item Otro aspecto en el que se podría ampliar la funcionalidad de Green In House es generar otro modelo totalmente automatizado e incluir actuadores que permitan realizar el riego automáticos o dar luz a la planta. Esto desviaría el proyecto de su principal enfoque educativo y su intención de obligar a los niños a interactuar los máximo posible con la planta, para aprender de este proceso. Por ello, si se quiere desarrollar, lo mejor sería desarrollar otro modelo independiente enfocado al cultivo automatizado.
\end{itemize}
