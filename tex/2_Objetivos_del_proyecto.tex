\capitulo{2}{Objetivos del proyecto}

Green In House es un proyecto que integra tecnología, ecología y educación de una manera práctica e interactiva, brindando a los niños la oportunidad de aprender sobre la importancia de cuidar el medio ambiente y asumir responsabilidades. Por lo tanto, Green In House no es solo una maceta, sino que es un instrumento de aprendizaje que promueve la comprensión de conceptos científicos como la fotosíntesis, la biología de las plantas y la interacción con el medio ambiente. Los sensores incorporados en la maceta ayudan a los niños a entender cómo diferentes factores como la luz, la humedad y la temperatura afectan el crecimiento de las plantas.

Además, la integración con una aplicación móvil proporciona una experiencia de aprendizaje más rica. La aplicación ofrece a los niños acceso a información detallada sobre las plantas, consejos de cuidado y la capacidad de registrar observaciones y hacer un seguimiento del crecimiento de sus plantas. Debido a esto es una herramienta educativa muy valiosa, ya que puede ayudar al aprendizaje infantil de manera significativa, vivencial y con aporte al aprendizaje académico de varias formas, a la vez que desarrolla nuevas habilidades trasversales gracias a la motivación:

\textbf{Aprendizaje significativo:} Green In House proporciona a los niños una experiencia práctica y tangible de cuidado de plantas, lo que les permite establecer conexiones significativas entre los conceptos teóricos y la realidad. Al participar activamente en el proceso de crecimiento de las plantas, los niños pueden comprender mejor los conceptos relacionados con la biología, la ecología, la fotosíntesis y otros temas relacionados.

\textbf{Aprendizaje vivencial:} Green In House ofrece una experiencia de aprendizaje práctica y tangible, donde los niños pueden interactuar directamente con las plantas y experimentar las consecuencias de sus acciones. Aprenden observando el crecimiento de las plantas, experimentando con diferentes condiciones ambientales y tomando decisiones sobre el cuidado de las mismas. Esta experiencia vivencial refuerza el aprendizaje al involucrar múltiples sentidos y brindar una experiencia multisensorial.

\textbf{Aporte al aprendizaje académico:} Green In House se puede integrar en el currículo académico, complementando y enriqueciendo las lecciones tradicionales. Las actividades relacionadas con las plantas pueden ser utilizadas para abordar diferentes áreas del conocimiento, como ciencias naturales, matemáticas (al medir el crecimiento de las plantas), lenguaje (a través de la comunicación y expresión de ideas sobre las plantas), arte (mediante la creación de proyectos artísticos inspirados en la naturaleza) y más. Esto amplía el alcance del aprendizaje académico y lo hace más contextualizado y relevante para los niños.

\textbf{Desarrollo de habilidades transversales:} Green In House promueve el desarrollo de habilidades transversales importantes, como la responsabilidad, el pensamiento crítico, la resolución de problemas, la toma de decisiones y la autonomía. Los niños aprenden a cuidar de las plantas, a tomar decisiones basadas en la observación y a solucionar problemas relacionados con el crecimiento de las mismas. Estas habilidades son transferibles a otras áreas de su vida y contribuyen a su desarrollo integral.

\textbf{Motivación y engagement:} El enfoque lúdico y práctico de Green In House captura el interés de los niños y los motiva a participar activamente en el aprendizaje. Al estar involucrados en el cuidado de las plantas, los niños se sienten más comprometidos y conectados con el proceso de aprendizaje, lo que favorece un mayor nivel de retención y comprensión de los contenidos.

Green In House, a pesar de estar basado en la tecnología e incorporar el uso de dispositivos móviles, está diseñado para promover la interacción con el mundo real y mejorar el desarrollo cognitivo de los niños. Algunas razones por las cuales este proyecto podría tener un impacto positivo en lugar de fomentar una dependencia negativa de la tecnología:

\textbf{Interacción con el mundo real}: Aunque Green In House usa una aplicación móvil, la mayoría de las acciones necesarias para el cuidado de la planta se realizan en el mundo físico. Los niños tienen que regar la planta, colocarla en un lugar donde reciba suficiente luz y monitorear los parámetros del ambiente físico. Esto significa que pasan un tiempo significativo interactuando con el mundo real, no solo con la tecnología.

\textbf{Consecuencias tangibles}: A diferencia de muchos juegos digitales, donde las acciones pueden no tener consecuencias reales, en Green In House las decisiones y acciones de los niños tienen resultados tangibles. Si olvidan regar su planta, verán cómo se marchita. Esto puede ayudarles a comprender que sus acciones tienen consecuencias en el mundo real.

\textbf{Fomenta la responsabilidad}: El cuidado regular de una planta puede enseñar a los niños sobre la responsabilidad y la gestión del tiempo. Tienen que recordar regar la planta y comprobar sus condiciones regularmente. Este es un hábito que puede transferirse a otras áreas de sus vidas.

\textbf{Uso de la tecnología con propósito}: Green In House no promueve el uso de la tecnología por el simple hecho de usarla. En cambio, la tecnología se utiliza como una herramienta para aprender y cuidar de algo vivo. Esto puede ayudar a los niños a entender que la tecnología tiene muchos usos útiles y beneficiosos, más allá del entretenimiento.

Por lo tanto, aunque Green In House integra la tecnología, lo hace de una manera que promueve la interacción positiva y la comprensión del mundo real. Al proporcionar una experiencia de aprendizaje práctica y consecuencias reales para las acciones, puede ayudar a los niños a desarrollar habilidades valiosas y a entender cómo sus acciones afectan al mundo que les rodea.

Green In House, a pesar de estar basado en la tecnología e incorporar el uso de dispositivos móviles, está diseñado para promover la interacción con el mundo real y mejorar el desarrollo cognitivo de los niños. Aquí hay algunas razones por las cuales este proyecto podría tener un impacto positivo en lugar de fomentar una dependencia negativa de la tecnología:

Interacción con el mundo real: Aunque Green In House usa una aplicación móvil, la mayoría de las acciones necesarias para el cuidado de la planta se realizan en el mundo físico. Los niños tienen que regar la planta, colocarla en un lugar donde reciba suficiente luz y monitorear los parámetros del ambiente físico. Esto significa que pasan un tiempo significativo interactuando con el mundo real, no solo con la tecnología.

Consecuencias tangibles: A diferencia de muchos juegos digitales, donde las acciones pueden no tener consecuencias reales, en Green In House las decisiones y acciones de los niños tienen resultados tangibles. Si olvidan regar su planta, verán cómo se marchita. Esto puede ayudarles a comprender que sus acciones tienen consecuencias en el mundo real.

Fomenta la responsabilidad: El cuidado regular de una planta puede enseñar a los niños sobre la responsabilidad y la gestión del tiempo. Tienen que recordar regar la planta y comprobar sus condiciones regularmente. Este es un hábito que puede transferirse a otras áreas de sus vidas.

Uso de la tecnología con propósito: Green In House no promueve el uso de la tecnología por el simple hecho de usarla. En cambio, la tecnología se utiliza como una herramienta para aprender y cuidar de algo vivo. Esto puede ayudar a los niños a entender que la tecnología tiene muchos usos útiles y beneficiosos, más allá del entretenimiento.

Por lo tanto, aunque Green In House integra la tecnología, lo hace de una manera que promueve la interacción positiva y la comprensión del mundo real. Al proporcionar una experiencia de aprendizaje práctica y consecuencias reales para las acciones, puede ayudar a los niños a desarrollar habilidades valiosas y a entender cómo sus acciones afectan al mundo que les rodea.


En resumen, Green In House contribuye al aprendizaje infantil significativo, vivencial y con aporte al aprendizaje académico al brindar experiencias prácticas y reales de cuidado de plantas, integrar conceptos teóricos en un contexto tangible, desarrollar habilidades transversales y promover la motivación y el engagement en el proceso de aprendizaje. Aunque pricipalmente está diseñado y pensado para niños, también puede ser útil para los adultos y hay varias segmentaciones de clientes que podrían encontrar este producto interesante:

Jardineros aficionados: Los adultos que disfrutan del jardín y quieren aprender más sobre el cuidado de las plantas podrían beneficiarse de las características interactivas y educativas de Green In House.

Amantes de la tecnología: Aquellos que están interesados en la tecnología, especialmente la tecnología que se integra con la vida cotidiana de formas nuevas e interesantes, pueden ver el valor en un producto como Green In House.

Profesores y educadores: Los profesores podrían utilizar Green In House como una herramienta educativa en las aulas para enseñar sobre ciencias naturales, ecología y biología, o para fomentar la responsabilidad y el cuidado del medio ambiente.

Padres: Los padres que quieren proporcionar experiencias educativas prácticas para sus hijos en casa podrían estar interesados en Green In House.

Personas que viven en espacios reducidos: Para aquellos que viven en apartamentos o casas sin jardín, Green In House ofrece una forma de tener y cuidar plantas en espacios interiores.

Empresas de bienestar en el trabajo: Las empresas que buscan mejorar el bienestar de sus empleados pueden estar interesadas en utilizar Green In House para animar a los empleados a tomar descansos activos, cuidar una planta y desconectar de su trabajo por un tiempo.

Asistentes de terapia ocupacional y psicólogos: Green In House podría ser utilizado como una herramienta en la terapia ocupacional o en el tratamiento de trastornos de atención, enseñando a los pacientes a concentrarse en tareas, a ser conscientes del momento presente y a cuidar de algo más.

Estos son solo algunos ejemplos de cómo Green In House podría ser útil para los adultos. Este producto tiene un amplio alcance debido a su combinación única de tecnología y educación práctica en el cuidado de las plantas.






Desglose detallado de cada objetivo de la introduccion

tambien objetivos personales de que quiero conseguir con el proyecto


Este apartado explica de forma precisa y concisa cuales son los objetivos que se persiguen con la realización del proyecto. Se puede distinguir entre los objetivos marcados por los requisitos del software a construir y los objetivos de carácter técnico que plantea a la hora de llevar a la práctica el proyecto.
