\capitulo{2}{Objetivos del proyecto}
Tal y como se ha comentado anteriormente, Green In House es un proyecto que trata de unir el uso de la tecnología y la interacción de las personas con el entorno que nos rodea. Debido a eso, este proyecto proyecto persigue tanto objetivos sociales como objetivos tecnológicos, a parte de mis propios objetivos personales, los cuales so describen en los siguientes apartados. 

\section{Objetivos sociales}\label{objetivos-sociales}
Green In House es un proyecto que integra tecnología, ecología y educación de una manera práctica e interactiva, brindando a los niños la oportunidad de aprender sobre la importancia de cuidar el medio ambiente y asumir responsabilidades. Por lo tanto, Green In House es un instrumento de aprendizaje que promueve la comprensión de conceptos científicos como la fotosíntesis, la biología de las plantas y la interacción con el medio ambiente. Los sensores incorporados en la maceta ayudan a los niños a entender cómo diferentes factores como la luz, la humedad y la temperatura afectan el crecimiento de las plantas.

Además, la integración con una aplicación móvil proporciona una experiencia de aprendizaje más completa. La aplicación ofrece a los niños acceso a información detallada sobre las plantas, consejos de cuidados y la capacidad hacer un seguimiento del crecimiento de sus plantas en base a los factores ambientales a los que están sometidas las plantas. Debido a esto, Green In House puede ser una herramienta educativa muy valiosa, ya que puede ayudar al aprendizaje infantil de manera significativa, vivencial y con aporte al aprendizaje académico de varias formas, a la vez que desarrolla nuevas habilidades transversales gracias a la motivación en distintos aspectos como:
\begin{itemize}
    \item \textbf{Aprendizaje significativo:} Green In House proporciona a los niños una experiencia práctica y tangible sobre el cuidado de las plantas. Esto permite a los niños establecer conexiones significativas entre los conceptos teóricos y la realidad. Al hacerles participar activamente en el proceso de crecimiento de las plantas, los niños pueden comprender mejor los conceptos relacionados con la biología, la ecología, la fotosíntesis y otros temas relacionados.
    \item \textbf{Aprendizaje vivencial:} mediante la obligación de los niños a interactuar directamente con las plantas y experimentar las consecuencias de sus acciones, Green In House consigue hacer que aprendan observando el crecimiento de las plantas, experimentando con diferentes condiciones ambientales y tomando decisiones sobre el cuidado de las mismas. Esta experiencia vivencial refuerza el aprendizaje.
    \item \textbf{Aporte al aprendizaje académico:} Green In House se puede integrar en el ámbito académico, complementando y enriqueciendo las lecciones tradicionales. Las actividades relacionadas con las plantas pueden ser utilizadas para abordar diferentes áreas del conocimiento, como ciencias naturales, lenguaje (a través de la comunicación y expresión de ideas sobre las plantas), arte (mediante la creación de proyectos artísticos inspirados en la naturaleza),... Esto amplía el alcance del aprendizaje académico y lo hace más relevante para los niños.
    \item \textbf{Desarrollo de habilidades transversales:} Green In House promueve el desarrollo de habilidades como la responsabilidad, el pensamiento crítico, la resolución de problemas, la toma de decisiones y la autonomía. Los niños aprenden a cuidar de las plantas, a tomar decisiones basadas en la observación y a solucionar problemas relacionados con el crecimiento de las mismas. Estas habilidades se pueden extrapolar a otras áreas de su vida y contribuyen a su desarrollo integral.
    \item \textbf{Motivación y compromiso:} mediante su enfoque lúdico y práctico, Green In House pretende capturar el interés de los niños y motivarles a participar activamente en el proceso. Al estar involucrados en el cuidado de las plantas, los niños se sentirán más comprometidos y conectados con el proceso de aprendizaje, lo que favorece un mayor nivel de retención y comprensión de los contenidos.
\end{itemize}
Green In House, a pesar de estar basado en la tecnología e incorporar el uso de sensores y dispositivos móviles, está diseñado para promover la interacción con el mundo real y mejorar el desarrollo cognitivo de los niños. Algunas razones por las cuales este proyecto podría tener un impacto positivo, en lugar de fomentar una dependencia negativa de la tecnología, son las siguientes:
\begin{itemize}
    \item \textbf{Interacción con el mundo real}: aunque Green In House usa sensores y una aplicación móvil, la totalidad de las acciones necesarias para el cuidado de la planta se realizan de manera manual. Los niños tienen que regar la planta, colocarla en un lugar donde reciba suficiente luz y monitorizar los parámetros del ambiente físico. Esto significa que pasan un tiempo significativo interactuando con el mundo real, no solo con la tecnología.
    \item \textbf{Consecuencias tangibles}: a diferencia de la mayoría de videojuegos, donde las acciones no suelen tener consecuencias reales (a excepción de realizar de compras con dinero real, lo cual afecta directamente a sus cuentas de crédito, o en este caso, a las de sus padres), en Green In House las decisiones y acciones de los niños tienen resultados tangibles. Si olvidan regar su planta, verán cómo se marchita. Esto puede ayudarles a comprender que sus acciones tienen consecuencias en el mundo real. 
    \item \textbf{Fomenta la responsabilidad}: cuidar regularmente de una planta puede enseñar a los niños sobre la responsabilidad y la gestión del tiempo. Tienen que recordar regar la planta y comprobar sus condiciones regularmente. Este es un hábito que puede extrapolarse a otras áreas de sus vidas.  
    \item \textbf{Uso de la tecnología con un determinado propósito}: Green In House no promueve el uso de la tecnología por el simple hecho de usarla. En este proyecto la tecnología se utiliza como una herramienta para aprender y cuidar de un organismo vivo. Esto puede ayudar a los niños a entender que la tecnología tiene muchos usos útiles y beneficiosos, más allá del entretenimiento.
\end{itemize}

\section{Objetivos técnicos}
A continuación se describen los objetivos técnicos cubiertos en el proyecto:
\begin{itemize}
    \item Generar y mantener un registro temporal de datos de temperatura, luminosidad y humedad ambiente, así como la humedad de la maceta. Estos son los factores principales de influencia en el desarrollo de una planta en los que se centrará Green In House.
    \item Utilizar Git como sistema de control de versiones junto con
    una plataforma de repositorios \textit{Open Source}.
    \item Crear uno o varios servidores en una Raspberry Pi capaz de desplegar y controlar la aplicación completa de Green In House.
    \item Generar una serie de \textit{scripts} en Bash que permitan realizar fácilmente la instalación de la aplicación en cualquier Raspberry Pi y su iniciación automática al encenderla.
    \item Leer diferentes sensores electrónicos que miden factores ambientales de temperatura, humedad y luminosidad.
    \item Desarrollar una base de datos que almacene toda la información recogida de los sensores y la interrelacione con la planta a la que dichos sensores están asociados a los largo de diferentes periodos temporales y a los consejos asociados a dicha planta.
    \item Generar una estructura modular en la que se puedan incorporar fácilmente y de manera dinámica, nuevos sensores, plantas, tipos de plantas y consejos de mantenimiento, así como definir interrelaciones entre ellos, sin necesidad de modificar el código de la aplicación.
    \item Desarrollar una aplicación para generar una interfaz gráfica que permita realizar determinadas acciones de control de la maceta desde una pantalla táctil incorporada a la misma. 
    \item Implementar un sistema API REST capaz de comunicar datos entre el servidor alojado en la Raspberry Pi y aplicaciones externas que hagan uso de ella.  
    \item Desarrollar una aplicación multiplataforma en Flutter capaz de ser desplegada en diferentes plataformas como Android, iOS,... y que permita leer los registros de los sensores y graficarlos, así como generar nuevas plantas, incorporar sensores, modificar los consejos, etc.   
    \item Utilizar patrones de diseño para mejorar la arquitectura y facilitar el diseño y entendimiento del código. Emplear un sistema MVC (Modelo-Vista-Controlador) definido por capas.   
    \item Aplicar metodología ágil en el desarrollo del software 
\end{itemize}

\section{Objetivos personales}
A continuación se describen mis objetivos personales que buscaba cubrir con la realización de este proyecto:
\begin{itemize}
    \item Investigar diferentes tecnologías existentes utilizadas comúnmente para desarrollar las tareas de:
    \begin{itemize}
        \item Gestión de bases de datos en Raspberry Pi
        \item Comunicación entre sistemas mediante API REST en Raspberry Pi
        \item Generación de aplicaciones de interfaz gráfica para Raspberry Pi.
        \item Realizar lecturas de diferentes sensores electrónicos desde una Raspberry Pi.   
        \item Generación mediante Flutter de aplicaciones móviles con comunicación con otros sistemas mediante API REST.
    \end{itemize}
    \item Aportar un enfoque lúdico al cuidados de plantas mediante el uso de nuevas tecnologías.
\end{itemize}