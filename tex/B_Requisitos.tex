\apendice{Especificación de Requisitos}
En este apartado se recogen los requisitos especificados por el cliente para la realización de Green In House. Como es un proyecto propio, el cliente soy yo mismo, que también soy el desarrollador, pero a la vez he contado con la ayuda de mis tutores para especificar los requisitos de Green In House.

\section{Introducción}
Al principio la especificación de requisitos era bastante simple, pero a medida que fui construyendo el código, se me fueron ocurriendo nuevas funcionalidades que implementar y al final la especificación de requisitos se ha extendido bastante.

\section{Objetivos generales}
%TODO
\section{Catalogo de requisitos}
%TODO
\section{Especificación de requisitos}
%TODO

% Caso de Uso 1 -> Consultar Experimentos.
\begin{table}[p]
	\centering
	\begin{tabularx}{\linewidth}{ p{0.21\columnwidth} p{0.71\columnwidth} }
		\toprule
		\textbf{CU-1}    & \textbf{Ejemplo de caso de uso}\\
		\toprule
		\textbf{Versión}              & 1.0    \\
		\textbf{Autor}                & Alumno \\
		\textbf{Requisitos asociados} & RF-xx, RF-xx \\
		\textbf{Descripción}          & La descripción del CU \\
		\textbf{Precondición}         & Precondiciones (podría haber más de una) \\
		\textbf{Acciones}             &
		\begin{enumerate}
			\def\labelenumi{\arabic{enumi}.}
			\tightlist
			\item Pasos del CU
			\item Pasos del CU (añadir tantos como sean necesarios)
		\end{enumerate}\\
		\textbf{Postcondición}        & Postcondiciones (podría haber más de una) \\
		\textbf{Excepciones}          & Excepciones \\
		\textbf{Importancia}          & Alta o Media o Baja... \\
		\bottomrule
	\end{tabularx}
	\caption{CU-1 Nombre del caso de uso.}
\end{table}