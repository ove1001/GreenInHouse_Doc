\apendice{Especificación de Requisitos}
En este apartado se recogen los requisitos especificados por el cliente para la realización de Green In House. Como es un proyecto propio, el cliente soy yo mismo, que también soy el desarrollador, pero a la vez he contado con la ayuda de mis tutores para especificar los requisitos de Green In House.

\section{Introducción}
Al principio la especificación de requisitos era bastante simple, pero a medida que fui construyendo el código, se me fueron ocurriendo nuevas funcionalidades que implementar y al final la especificación de requisitos se ha extendido bastante.

\section{Objetivos generales}
A continuación se describen los objetivos generales del proyecto:
\begin{itemize}
    \item Leer diferentes sensores electrónicos que miden factores físicos como la temperatura, humedad y luminosidad de diferentes zonas como la maceta o el ambiente.
    \item Desarrollar una base de datos que almacene toda la información recogida de los sensores y la interrelacione con la planta a la que dichos sensores están asociados a los largo de diferentes periodos temporales y a los consejos asociados a dicha planta. Esto permitirá mantener un registro temporal de datos de temperatura, luminosidad y humedad ambiente, así como la humedad de la maceta. Estos son los factores principales de influencia en el desarrollo de una planta en los que se centrará Green In House.
    \item Generar una estructura modular en la que se puedan incorporar fácilmente y de manera dinámica y persistente, nuevos sensores, plantas, tipos de plantas y consejos de mantenimiento, así como definir interrelaciones entre ellos, sin necesidad de modificar el código de la aplicación.
    \item Desarrollar una aplicación para generar una interfaz gráfica que permita realizar determinadas acciones de control de la maceta desde una pantalla táctil incorporada a la misma. Entre estas acciones se encuentra la de permitir al usuario introducir las credenciales de su red WiFi, para permitir a la maceta comunicarse por red.
    \item Implementar un sistema API REST capaz de comunicar datos por red entre el servidor alojado en la Raspberry Pi y aplicaciones externas que hagan uso de ella. Por medio de esta API REST se podrá crear en el sistema nuevos sensores, tipos de plantas, plantas y consejos asociados a ellas.
    \item Desarrollar una aplicación móvil en Flutter capaz de ser desplegada en diferentes plataformas como Android, iOS,... y que permita leer los registros de los sensores y graficarlos, así como generar nuevas plantas, incorporar sensores, modificar los consejos, etc. haciendo uso del servidor API REST.
    \item Generar una serie de \textit{scripts} en Bash que permitan realizar fácilmente la instalación de la aplicación de Green In House en cualquier Raspberry Pi y su iniciación automática al encenderla.
\end{itemize}

\section{Catalogo de requisitos}
A continuación se enumeran los requisitos funcionales y no funcionales derivados de los objetivos generales del proyecto.

    \subsection{Requisitos no funcionales}
    \begin{itemize}
        \item \textbf{RNF-1 Utilización de Raspberry:} Se utilizará una Raspberry Pi para leer los sensores, almacenar los valores y gestionar la comunicación y tratamiento de los datos.
        \item \textbf{RNF-2 Almacenamiento de los datos:} el almacenamiento de los datos de manera persistente se hará utilizando una base de datos y su correspondiente gestor.
        \begin{itemize}
            \item \textbf{RNF-1.1 Borrado de los datos:} Para no violar la integridad referencial de los datos de la base de datos, no se permite borrar las instancias de los datos una vez grabadas. Para poder marcarlas como eliminadas se ha empleado un sistema de fechas, el cual determina cuando se creó la instancia y cuando se declaró como eliminada.
        \end{itemize}
        \item \textbf{RNF-2 Utilización de Python} el código de la aplicación para la Raspberry Pi será desarrolado en Python para agilizar el desarrollo del prototipo.
        \item \textbf{RNF-3 Utilización de Flutter} el código de la aplicación movil será desarrolado en Flutter al ser de desarrollo multiplataforma, para poder desplegar la misma aplicación en varios sistemas.
        \item \textbf{RNF-4 Comunicación de datos con otros sistemas} para que otros dispositivos puedan hacer uso de los datos recogidos por Green In House se utilizará conetividad red mediante un servidor API REST.
        \item \textbf{RNF-5 Conexión a red WiFi} será necesario suministrar al usuario una manera de introducir en el sistema las credenciales de la red WiFi.
    \end{itemize}

    \subsection{Requisitos funcionales}
    \begin{itemize}
        \item \textbf{RF-1 Gestión de sensores:} la aplicación tiene que ser capaz de gestionar los sensores del sistema.
            \begin{itemize}
                \item \textbf{RF-1.1 Creación de sensores:} la aplicación tiene que ser capaz de crear nuevos sensores en el sistema, especificando la zona en la que estará ubicado el sensor, el tipo de medición que realizará, el modelo de sensor que se utilizará y los pines y dirección que se utilizarán para realizar su lectura. Para facilitar su posterior identificación por parte del usuario se le asignará también un nombre y se almacenará la fecha en la que se dio de alta el sensor.
                \item \textbf{RF-1.2 Modificación de sensores:} la aplicación tiene que ser capaz de modificar los datos de los sensores existentes en el sistema.
                \item \textbf{RF-1.3 Eliminación de sensores:} la aplicación tiene que ser capaz de eliminar sensores existentes en el sistema. 
                \item \textbf{RF-1.4 Obtención de sensores:} la aplicación tiene que ser capaz de recuperar los datos de los sensores existentes en el sistema.
                \begin{itemize}
                    \item \textbf{RF-1.4.1 Obtención de sensores activos:} la aplicación tiene que ser capaz de recuperar los datos de los sensores activos existentes en el sistema filtrándoles por su fecha de baja. Este filtrado se tiene que poder aplicar a todos los filtrados detallados continuación. 
                    \item \textbf{RF-1.4.2 Obtención de sensores por tipo:} la aplicación tiene que ser capaz de recuperar los datos de los sensores existentes en el sistema filtrándoles por su tipo de medición.
                    \item \textbf{RF-1.4.3 Obtención de sensores por zona:} la aplicación tiene que ser capaz de recuperar los datos de los sensores existentes en el sistema filtrándoles por su zona de ubicación.
                    \item \textbf{RF-1.4.4 Obtención de sensores por tipo y zona:} la aplicación tiene que ser capaz de recuperar los datos de los sensores existentes en el sistema filtrándoles por su tipo de medición y su zona de ubicación.
                    \item \textbf{RF-1.4.5 Obtención de sensores por modelo:} la aplicación tiene que ser capaz de recuperar los datos de los sensores existentes en el sistema filtrándoles por su modelo.
                \end{itemize}
            \end{itemize}
            
        \item \textbf{RF-2 Gestión de plantas:} la aplicación tiene que ser capaz de gestionar las plantas del sistema.
            \begin{itemize}
                \item \textbf{RF-2.1 Creación de plantas:} la aplicación tiene que ser capaz de crear nuevas plantas en el sistema, especificando el tipo de planta y el nombre que se asignará a dicha planta.
                \item \textbf{RF-2.2 Modificación de plantas:} la aplicación tiene que ser capaz de modificar los datos de las plantas existentes en el sistema.
                \item \textbf{RF-2.3 Eliminación de plantas:} la aplicación tiene que ser capaz de eliminar plantas existentes en el sistema.
                \item \textbf{RF-2.4 Obtención de plantas:} la aplicación tiene que ser capaz de recuperar los datos de las plantas existentes en el sistema.
                \begin{itemize}
                    \item \textbf{RF-2.4.1 Obtención de plantas activas:} la aplicación tiene que ser capaz de recuperar los datos de las plantas activas existentes en el sistema filtrándolas por su fecha de baja. Este filtrado se tiene que poder aplicar a todos los filtrados detallados continuación. 
                    \item \textbf{RF-2.4.2 Obtención de plantas por tipo:} la aplicación tiene que ser capaz de recuperar los datos de las plantas existentes en el sistema filtrándolas por su tipo.
                \end{itemize}
            \end{itemize}
            
        \item \textbf{RF-3 Gestión de tipos de plantas:} la aplicación tiene que ser capaz de gestionar los tipos de plantas del sistema.
            \begin{itemize}
                \item \textbf{RF-3.1 Creación de tipos de plantas:} la aplicación tiene que ser capaz de crear nuevas tipos de plantas en el sistema, especificando el tipo de planta y el nombre que se asignará a dicha planta.
                \item \textbf{RF-3.2 Modificación de tipos de plantas:} la aplicación tiene que ser capaz de modificar los datos de las tipos de plantas existentes en el sistema.
                \item \textbf{RF-3.3 Obtención de tipos de plantas:} la aplicación tiene que ser capaz de recuperar los datos de las tipos de plantas existentes en el sistema.
            \end{itemize}
            
        %TODO RF-4 acociación entre sensor y planta
        \item \textbf{RF-4 Gestión de asociaciones entre sensores y plantas:} la aplicación tiene que ser capaz de gestionar las asociaciones entre los sensores y las plantas del sistema.
            \begin{itemize}
                \item \textbf{RF-4.1 Creación de asociación sensor-planta:} la aplicación tiene que ser capaz de crear nuevas asociaciones entre los sensores y las plantas existentes en el sistema, especificando el sensor y la planta asociados que se quieren asociar.
                \item \textbf{RF-4.2 Eliminación de asociación sensor-planta:} la aplicación tiene que ser capaz de eliminar las asociaciones existentes entre los sensores y las plantas del sistema, especificando el sensor y la planta asociados que se quieren desasociar.
                \item \textbf{RF-4.3 Obtención de asociación sensor-planta:} la aplicación tiene que ser capaz de obtener las asociaciones existentes entre los sensores y las plantas del sistema, especificando el sensor y la planta asociados que se quiere modificar.
                \begin{itemize}
                    \item \textbf{RF-4.3.1 Obtención de asociaciones sensor-planta activas:} la aplicación tiene que ser capaz de recuperar las asociaciones activas entre sensores y plantas existentes en el sistema filtrándolas por su fecha de baja. Este filtrado se tiene que poder aplicar a todos los filtrados detallados continuación. 
                    \item \textbf{RF-4.3.2 Obtención de todos los sensores asociados a una planta:} la aplicación tiene que ser capaz de obtener todos los sensores asociados a una planta del sistema, especificando la planta de la cual se quieren obtener los sensores.
                    \item \textbf{RF-4.3.3 Obtención de todas las plantas asociadas a un sensor:} la aplicación tiene que ser capaz de obtener todas las plantas asociadas a un sensor del sistema, especificando el sensor de la cual se quieren obtener las plantas.
                \end{itemize}
            \end{itemize}
        
        \item \textbf{RF-5 Gestión de registros de sensores:} la aplicación tiene que ser capaz de gestionar los registros de sensores del sistema.
            \begin{itemize}
                \item \textbf{RF-5.1 Creación de registros de sensores:} la aplicación tiene que ser capaz de crear nuevos registros de sensores en el sistema, especificando el sensor que generó el registro,  el valor del sensor y la unidad de medida del registro.
                \item \textbf{RF-5.2 Obtención de registros de sensores:} la aplicación tiene que ser capaz de recuperar los datos de los registros de sensores existentes en el sistema.
                \begin{itemize}
                    \item \textbf{RF-5.2.1 Obtención de registros de sensores generados entre determinadas fechas:} la aplicación tiene que ser capaz de recuperar los datos de los registros de sensores existentes en el sistema filtrándoles por su fecha de creación. Este filtrado se tiene que poder aplicar a todos los filtrados detallados continuación. 
                    \item \textbf{RF-5.2.2 Obtención de registros de sensores en formato para graficar:} la aplicación tiene que ser capaz de recuperar los datos de los registros de sensores existentes en el sistema agrupandoles en dos listas (fechas para el eje X y valores para el eje Y) para poder graficarles fácilmente. Este filtrado se tiene que poder aplicar a todos los filtrados detallados continuación. 
                    \item \textbf{RF-5.2.3 Obtención de registros de sensores por sensor:} la aplicación tiene que ser capaz de recuperar los datos de los registros de sensores existentes en el sistema filtrándoles por el sensor que generó dicho registro.
                    \item \textbf{RF-5.2.4 Obtención de registros de sensores por planta:} la aplicación tiene que ser capaz de recuperar los datos de los registros de sensores existentes en el sistema filtrándoles por la planta a la que están asociados.
                \end{itemize}
            \end{itemize}
            
        \item \textbf{RF-6 Gestión de consejos de plantas:} la aplicación tiene que ser capaz de gestionar los consejos de plantas del sistema.
            \begin{itemize}
                \item \textbf{RF-6.1 Creación de consejos de plantas:} la aplicación tiene que ser capaz de crear nuevos consejos de plantas en el sistema, el tipo de medida, la zona de ubicación,  la unidad de medida y los valores mínimos y máximos.
                \item \textbf{RF-6.2 Modificación de consejos de plantas:} la aplicación tiene que ser capaz de modificar los datos de los consejos de plantas existentes en el sistema.
                \item \textbf{RF-6.3 Obtención de consejos de plantas:} la aplicación tiene que ser capaz de recuperar los datos de los consejos de plantas existentes en el sistema.
                \begin{itemize}
                    \item \textbf{RF-6.3.1 Obtención de consejos de plantas por planta:} la aplicación tiene que ser capaz de recuperar los datos de los consejos de plantas existentes en el sistema filtrándoles por la planta a la que están asociados. Este filtrado se tiene que poder aplicar a todos los filtrados detallados continuación. 
    				\item \textbf{RF-6.3.2 Obtención de consejos de plantas por zona:} la aplicación tiene que ser capaz de recuperar los datos de los consejos de plantas existentes en el sistema filtrándoles por la zona a la que están asociados.
    				\item \textbf{RF-6.3.3 Obtención de consejos de plantas por tipo de medida:} la aplicación tiene que ser capaz de recuperar los datos de los consejos de plantas existentes en el sistema filtrándoles por el tipo de medida a la que están asociados.
                \end{itemize}
            \end{itemize}
            
        \item \textbf{RF-7 Gestión de consejos de tipos de plantas:} la aplicación tiene que ser capaz de gestionar los consejos de tipos de plantas del sistema.
            \begin{itemize}
                \item \textbf{RF-7.1 Creación de consejos de tipos de plantas:} la aplicación tiene que ser capaz de crear nuevos consejos de tipos de plantas en el sistema, el tipo de medida, la zona de ubicación,  la unidad de medida y los valores mínimos y máximos.
                \item \textbf{RF-7.2 Modificación de consejos de tipos de plantas:} la aplicación tiene que ser capaz de modificar los datos de los consejos de tipos de plantas existentes en el sistema.
                \item \textbf{RF-7.3 Obtención de consejos de tipos de plantas:} la aplicación tiene que ser capaz de recuperar los datos de los consejos de tipos de plantas existentes en el sistema.
                \begin{itemize}
                    \item \textbf{RF-7.3.1 Obtención de consejos de tipos de plantas por planta:} la aplicación tiene que ser capaz de recuperar los datos de los consejos de tipos de plantas existentes en el sistema filtrándoles por el tipo de planta a la que están asociados. Este filtrado se tiene que poder aplicar a todos los filtrados detallados continuación. 
    				\item \textbf{RF-7.3.2 Obtención de consejos de tipos de plantas por zona:} la aplicación tiene que ser capaz de recuperar los datos de los consejos de tipos de plantas existentes en el sistema filtrándoles por la zona a la que están asociados.
    				\item \textbf{RF-7.3.3 Obtención de consejos de tipos de plantas por tipo de medida:} la aplicación tiene que ser capaz de recuperar los datos de los consejos de tipos de plantas existentes en el sistema filtrándoles por el tipo de medida a la que están asociados.
                \end{itemize}
            \end{itemize}
            %TODO
        \item \textbf{RF-8 Lectura de sensores electrónicos:} Se tienen que poder leer sensores electrónicos conectados al sistema, utilizando los datos almacenados de los sensores y generando registros de sesores con los valores leidos.
        \item \textbf{RF-9 Conexión a red WiFi:} Se tiene que poder introducir las credenciales de la red WiFi a conectar.
        \item \textbf{RF-10 Graficar los datos de lo sensores desde una aplicación móvil:} Se tiene que poder graficar desde una aplicación móvil los registros de los sensores agrupándoles en diferentes periodos temporales.
    \end{itemize}

\newpage
\section{Especificación de requisitos}

Este caso de uso es aplicable a todas las entidades del sistema. Únicamente cambiarían los datos suministrados para realizar la creación de la instancia de la entidad respectiva y variaría el tipo de objeto generado, correspondiente al tipo de entidad creada, así como los datos del archivo JSON devuelto al cliente.

\begin{longtable}{ p{0.21\columnwidth} p{0.71\columnwidth} }
    \caption{CU-1 Creación de una entidad.}\\
    \hline
    \textbf{CU-1}    & \textbf{Creación de una instancia de una entidad}\\
    \hline
    \endfirsthead
    \hline
    \textbf{CU-1}    & \textbf{Creación de una instancia de una entidad}\\
    \hline
    \endhead
    \hline
    \multicolumn{2}{c}{Sigue en la página siguiente.}
    \endfoot
    \hline
    \endlastfoot

  \textbf{Versión}              & 1.0    \\
    \textbf{Autor}                & Oscar Valverde Escobar \\
    \textbf{Requisitos asociados} & RF-1.1, RF-2.1, RF-3.1, RF-4.1, RF-5.1, RF-6.1, RF-7.1\\
    \textbf{Descripción}          & Operaciones realizadas para crear una nueva instancia de una entidad de manera persistente en la base de datos de la aplicación.\\
    \textbf{Precondición}         & 
    \begin{itemize}
        \item El sistema tiene que estar conectado a la red WiFi y tener el servidor API REST funcionando. 
        \item La instancia de la entidad existe en la base de datos. 
        \item Se dispone de todos los datos necesarios para eliminar la entidad.
        \item Si contiene referencias a otras entidades, tienen que existir dichas instancias de las entidades en la base de datos.
        \item El almacenamiento del sistema no está lleno
    \end{itemize}\\
    \textbf{Acciones}             &
    \begin{enumerate}
        \def\labelenumi{\arabic{enumi}.}
        \tightlist
        \item Suministrar al \textit{endpoint} de la API REST los datos necesarios para crear la entidad correspondiente .
        \item Verificar que dichos datos son válidos.
        \item Comunicar la petición de la API REST a la capa de servicios para que resuelva como realizar la creación.
        \item Comunicar la petición de la capa de servicios al manejador de la base de datos para que resuelva como realizar la creación.
        \item Realizar la creación del objeto en la base de datos.
        \item Devolución del objeto creado por el manejador de la base de datos a la capa de servicios.
        \item Devolución del objeto creado por la capa de servicios a la API REST.
        \item Devolución  al cliente  desde la API REST de un archivo JSON con los atributos del objeto creado por la capa de servicios junto con un código HTML de estado de la operación. 
    \end{enumerate}\\
    \textbf{Postcondición}        & Existirá una nueva instancia de entidad en la tabla correspondiente de la base de datos. \\
    \textbf{Excepciones}          & 
        \begin{enumerate}
            \item EntidadExiste
            \item Código HTML de error
        \end{enumerate} \\
    \textbf{Importancia}          & Alta \\
    \bottomrule
    \hline
\end{longtable}


\newpage
Este caso de uso es aplicable a todas las entidades del sistema. Únicamente cambiarían los datos suministrados para realizar la modificación de la instancia de la entidad respectiva y variaría el tipo de objeto generado, correspondiente al tipo de entidad modificado, así como los datos del archivo JSON devuelto al cliente.

\begin{longtable}{ p{0.21\columnwidth} p{0.71\columnwidth} }
    \caption{CU-2 Modificación de una entidad.}\\
    \hline
    \textbf{CU-2}    & \textbf{Modificación de una instancia de una entidad}\\
    \hline
    \endfirsthead
    \hline
    \textbf{CU-2}    & \textbf{Modificación de una instancia de una entidad}\\
    \hline
    \endhead
    \hline
    \multicolumn{2}{c}{Sigue en la página siguiente.}
    \endfoot
    \hline
    \endlastfoot
    
    \textbf{Versión}              & 1.0    \\
    \textbf{Autor}                & Oscar Valverde Escobar \\
    \textbf{Requisitos asociados} & RF-1.2, RF-2.2, RF-3.2, RF-6.2, RF-7.2\\
    \textbf{Descripción}          & Operaciones realizadas para modificar una instancia de entidad de manera persistente en la base de datos de la aplicación.\\
    \textbf{Precondición}         & 
    \begin{itemize}
        \item El sistema tiene que estar conectado a la red WiFi y tener el servidor API REST funcionando. 
        \item La instancia de la entidad tiene que existir en la base de datos. 
        \item Se dispone de todos los datos necesarios para obtener y modificar la entidad.
        \item Si contiene referencias a otras entidades, tienen que existir dichas instancias de las entidades en la base de datos.
    \end{itemize}\\
    \textbf{Acciones}             &
    \begin{enumerate}
        \def\labelenumi{\arabic{enumi}.}
        \tightlist
        \item Suministrar al \textit{endpoint} de la API REST los datos necesarios para recuperar y modificar la entidad correspondiente.
        \item Verificar que dichos datos son válidos.
        \item Comunicar la petición de la API REST a la capa de servicios para que resuelva como realizar la modificación de la instancia de la entidad.
        \item Comunicar la petición de la capa de servicios al manejador de la base de datos para que resuelva como realizar la recuperación y modificación de la instancia de la entidad.
        \item Realizar la obtención de la instancia de la entidad de la base de datos y generar su objeto correspondiente.
        \item Realizar la modificación del objeto generado.
        \item Realizar el grabado en la base de datos del objeto modificado.
        \item Devolución del objeto modificado por el manejador de la base de datos a la capa de servicios.
        \item Devolución del objeto modificado por la capa de servicios a la API REST.
        \item Devolución  al cliente  desde la API REST de un archivo JSON con los atributos del objeto modificado por la capa de servicios junto con un código HTML de estado de la operación. 
    \end{enumerate}\\
    \textbf{Postcondición}        & 
        \begin{itemize}
            \item Existirá el mismo número de instancias de entidad en la tabla correspondiente de la base de datos. 
            \item Los datos de la instancia especificada de la entidad habrán sido modificados en la tabla correspondiente de la base de datos.
        \end{itemize}\\
    \textbf{Excepciones}          & 
        \begin{enumerate}
            \item EntidadNoExiste
            \item Código HTML de error
        \end{enumerate} \\
    \textbf{Importancia}          & Alta \\
    \hline
\end{longtable}


\newpage
Este caso de uso es aplicable a todas las entidades del sistema. Únicamente cambiarían los datos suministrados para realizar la eliminación de la instancia de la entidad respectiva y variaría el tipo de objeto generado, correspondiente al tipo de entidad eliminado, así como los datos del archivo JSON devuelto al cliente.

\begin{longtable}{ p{0.21\columnwidth} p{0.71\columnwidth} }
    \caption{CU-3 Eliminación de una entidad.}\\
    \hline
    \textbf{CU-3}    & \textbf{Eliminación de una instancia de una entidad}\\
    \hline
    \endfirsthead
    \hline
    \textbf{CU-3}    & \textbf{Eliminación de una instancia de una entidad}\\
    \hline
    \endhead
    \hline
    \multicolumn{2}{c}{Sigue en la página siguiente.}
    \endfoot
    \hline
    \endlastfoot
    
    \textbf{Versión}              & 1.0    \\
    \textbf{Autor}                & Oscar Valverde Escobar \\
    \textbf{Requisitos asociados} & RF-1.3, RF-2.3, RF-4.2\\
    \textbf{Descripción}          & Operaciones realizadas para dar de baja una instancia de una entidad de manera persistente en la base de datos de la aplicación.\\
    \textbf{Precondición}         & 
        \begin{itemize}
            \item El sistema tiene que estar conectado a la red WiFi y tener el servidor API REST funcionando. 
            \item La instancia de la entidad tiene que existir en la base de datos. 
            \item Se dispone de todos los datos necesarios para dar de baja la entidad.
            \item Si contiene referencias a otras entidades, tienen que existir dichas instancias de las entidades en la base de datos.
        \end{itemize}\\
    \textbf{Acciones}             &
    \begin{enumerate}
        \def\labelenumi{\arabic{enumi}.}
        \tightlist
        \item Suministrar al \textit{endpoint} de la API REST los datos necesarios eliminar la entidad correspondiente.
        \item Verificar que dichos datos son válidos.
        \item Comunicar la petición de la API REST a la capa de servicios para que resuelva como realizar la eliminación de la instancia de la  entidad.
        \item Comunicar la petición de la capa de servicios al manejador de la base de datos para que resuelva como realizar la eliminación de la instancia de la  entidad.
        \item Realizar la obtención de la instancia de la entidad de la base de datos y generar su objeto correspondiente.
        \item Dar de baja la instancia de la entidad en la base de datos asignando al objeto generado la fecha en la que se realizo esta operación.
        \item Realizar el grabado en la base de datos del objeto modificado.
        \item Devolución del objeto modificado por el manejador de la base de datos a la capa de servicios.
        \item Devolución del objeto modificado por la capa de servicios a la API REST.
        \item Devolución  al cliente  desde la API REST de un archivo JSON con los atributos del objeto dado de baja por la capa de servicios junto con un código HTML de estado de la operación. 
    \end{enumerate}\\
    \textbf{Postcondición}        & 
        \begin{itemize}
            \item Existirá el mismo número de instancias de entidad en la tabla correspondiente de la base de datos. 
            \item Los datos de fecha de baja de la instancia especificada de la entidad habrán sido modificados en la tabla correspondiente de la base de datos.
        \end{itemize}\\
    \textbf{Excepciones}          & 
        \begin{enumerate}
            \item EntidadNoExiste
            \item Código HTML de error
        \end{enumerate} \\
    \textbf{Importancia}          & Alta \\
    \hline
\end{longtable}


\newpage
Este caso de uso es aplicable a todas las entidades del sistema. Únicamente cambiarían los datos suministrados para realizar la modificación de la instancia de la entidad respectiva y variaría el tipo de objeto generado, correspondiente al tipo de entidad modificado, así como los datos del archivo JSON devuelto al cliente.

\begin{longtable}{ p{0.21\columnwidth} p{0.71\columnwidth} }
    \caption{CU-4 Obtención de una entidad.}\\
    \hline
    \textbf{CU-4}    & \textbf{Obtención de una entidad}\\
    \hline
    \endfirsthead
    \hline
    \textbf{CU-4}    & \textbf{Obtención de una entidad}\\
    \hline
    \endhead
        \hline
        \multicolumn{2}{c}{Sigue en la página siguiente.}
        \endfoot
        \hline
        \endlastfoot
        
    \textbf{Versión}              & 1.0    \\
    \textbf{Autor}                & Oscar Valverde Escobar \\
    \textbf{Requisitos asociados} & RF-1.4, RF-2.4, RF-3.3, RF-4.3, RF-5.2, RF-6.3, RF-7.3\\
    \textbf{Descripción}          & Operaciones realizadas para obtener una entidad de la base de datos de la aplicación.\\
    \textbf{Precondición}         & 
        \begin{itemize}
            \item El sistema tiene que estar conectado a la red WiFi y tener el servidor API REST funcionando. 
            \item La instancia de la entidad existe en la base de datos. 
            \item Se dispone de todos los datos necesarios para eliminar la entidad.
            \item Si contiene referencias a otras entidades, tienen que existir dichas entidades en la base de datos.
        \end{itemize}\\
    \textbf{Acciones}             &
    \begin{enumerate}
        \def\labelenumi{\arabic{enumi}.}
        \tightlist
        \item Suministrar al \textit{endpoint} de la API REST los datos necesarios para obtener la instancia de la entidad correspondiente .
        \item Verificar que dichos datos son válidos.
        \item Comunicar la petición de la API REST a la capa de servicios para que resuelva como realizar la obtención de la instancia de la entidad.
        \item Comunicar la petición de la capa de servicios al manejador de la base de datos para que resuelva como realizar la recuperación de la instancia.
        \item Realizar la recuperación de la instancia de la entidad en la base de datos.
        \item Devolución del objeto recuperado por el manejador de la base de datos a la capa de servicios.
        \item Devolución del objeto recuperado por la capa de servicios a la API REST.
        \item Devolución  al cliente  desde la API REST de un archivo JSON con los atributos del objeto obtenido por la capa de servicios junto con un código HTML de estado de la operación. 
    \end{enumerate}\\
    \textbf{Postcondición}        & 
    \begin{itemize}
            \item Existirá el mismo número de instancias de entidad en la tabla correspondiente de la base de datos. 
            \item Los datos de la instancia especificada de la entidad no habrán variado en la tabla correspondiente de la base de datos.
            \item Se habrá devuelto un archivo JSON con los datos del objeto correspondiente
    \end{itemize}\\
    \textbf{Excepciones}          & 
        \begin{enumerate}
        \item EntidadNoExiste
            \item Código HTML de error
        \end{enumerate} \\
    \textbf{Importancia}          & Alta \\
    \hline
\end{longtable}

\newpage
Este caso de uso es aplicable a todas las entidades del sistema. Únicamente cambiarían los datos suministrados para realizar la modificación de la instancia de la entidad respectiva y variaría el tipo de objeto generado, correspondiente al tipo de entidad modificado, así como los datos del archivo JSON devuelto al cliente. Este caso de uso es una extensión de la funcionalidad del caso de uso 4, permitiendo recuperar varias instancias de una entidad, en lugar de una única instancia, filtrando las instancias a recuperar por alguno de sus atributos.

\begin{longtable}{ p{0.21\columnwidth} p{0.71\columnwidth} }
    \caption{CU-5 Obtención de varias instancias de una entidad filtrando por atributos.}\\
    \hline
    \textbf{CU-5}    & \textbf{Obtención de varias instancias de una entidad filtrando por atributos}\\
    \hline
    \endfirsthead
    \hline
    \textbf{CU-5}    & \textbf{Obtención de varias instancias de una entidad filtrando por atributos}\\
    \hline
    \endhead
        \hline
        \multicolumn{2}{c}{Sigue en la página siguiente.}
        \endfoot
        \hline
        \endlastfoot
        
    \textbf{Versión}              & 1.0    \\
    \textbf{Autor}                & Oscar Valverde Escobar \\
    \textbf{Requisitos asociados} & RF-1.4.1, RF-1.4.2, RF-1.4.3, RF-1.4.4, RF-1.4.5, RF-2.4.1, RF-2.4.2, RF-4.3.1, RF-4.3.2, RF-4.3.3, RF-5.2.1, RF-5.2.2, RF-5.2.3, RF-5.2.4, RF-6.3.1, RF-6.3.2, RF-6.3.3, RF-7.3.1, RF-7.3.2, RF-7.3.3\\
    \textbf{Descripción}          & Operaciones realizadas para obtener una entidad de la base de datos de la aplicación.\\
    \textbf{Precondición}         & 
        \begin{itemize}
            \item El sistema tiene que estar conectado a la red WiFi y tener el servidor API REST funcionando. 
            \item Existe en la base de datos instancias de las entidades que cumplen las cumplen los filtros establecidos. 
            \item Se dispone de todos los datos necesarios para obtener las entidades.
            \item Si contiene referencias a otras entidades, tienen que existir dichas entidades en la base de datos.
        \end{itemize}\\
    \textbf{Acciones}             &
    \begin{enumerate}
        \def\labelenumi{\arabic{enumi}.}
        \tightlist
        \item Suministrar al \textit{endpoint} de la API REST los datos necesarios para obtener las instancias de la entidad correspondiente (entidad a recuperar y filtros a aplicar).
        \item Verificar que dichos datos son válidos.
        \item Comunicar la petición de la API REST a la capa de servicios para que resuelva como realizar la obtención de las instancias de la entidad correspondiente a recuperar aplicando los filtros establecidos.
        \item Comunicar la petición de la capa de servicios al manejador de la base de datos para que resuelva como realizar la recuperación de las instancias de la entidad correspondiente aplicando los filtros establecidos.
        \item Realizar la recuperación de las instancias de la entidad correspondiente en la base de datos filtros establecidos.
        \item Devolución de la lista de objetos recuperados por el manejador de la base de datos a la capa de servicios.
        \item Devolución de la lista de objetos recuperados por la capa de servicios a la API REST.
        \item Devolución  al cliente  desde la API REST de un archivo JSON con los atributos de la lista de objetos obtenidos por la capa de servicios junto con un código HTML de estado de la operación. 
    \end{enumerate}\\
    \textbf{Postcondición}        &
    \begin{itemize}
            \item Existirá el mismo número de instancias de entidad en la tabla correspondiente de la base de datos. 
            \item Los datos de la instancia especificada de la entidad no habrán variado en la tabla correspondiente de la base de datos.
            \item Se habrá devuelto un archivo JSON con los datos del objeto correspondiente
    \end{itemize}\\
    \textbf{Excepciones}          & 
        \begin{enumerate}
        \item EntidadNoExiste
            \item Código HTML de error
        \end{enumerate} \\
    \textbf{Importancia}          & Media \\
    \hline
\end{longtable}


\begin{longtable}{ p{0.21\columnwidth} p{0.71\columnwidth} }
    \caption{CU-6 Lectura y registro de un sensor electrónico}\\
    \hline
    \textbf{CU-6}    & \textbf{Lectura y registro de un sensor electrónico}\\
    \hline
    \endfirsthead
    \hline
    \textbf{CU-6}    & \textbf{Lectura y registro de un sensor electrónico}\\
    \hline
    \endhead
        \hline
        \multicolumn{2}{c}{Sigue en la página siguiente.}
        \endfoot
        \hline
        \endlastfoot
    \textbf{Versión}              & 1.0    \\
    \textbf{Autor}                & Oscar Valverde Escobar \\
    \textbf{Requisitos asociados} & RF-1.4, RF-5.1, RF-8\\
    \textbf{Descripción}          & Operaciones realizadas para obtener un sensor de la base de datos, leer su valor real .\\
    \textbf{Precondición}         & 
        \begin{itemize}
            \item El sistema tiene que tener los sensores conectados correctamente.
            \item El sistema tiene que estar encendido
            \item Existe en la base de datos instancias de los sensores conectado con todos sus datos. 
            \item Se dispone de todos los datos necesarios para obtener el sensor.
            \item El almacenamiento del sistema no está lleno.
        \end{itemize}\\
    \textbf{Acciones}             &
    \begin{enumerate}
        \def\labelenumi{\arabic{enumi}.}
        \tightlist
        \item Se solicita al sistema obtener los datos de un sensor.
        \item Se realiza la lectura de dicho sensor
        \item Se genera un registro del valor leído asociado al sensor 
        \item Se almacena de manera persistente el registro generado
    \end{enumerate}\\
    \textbf{Postcondición}        &
    \begin{itemize}
            \item Existira una instancia nueva en la tabla de registros de sensores con los valores recogidos.
    \end{itemize}\\
    \textbf{Excepciones}          & 
        \begin{enumerate}
        \item SensorNoExiste
        \end{enumerate} \\
    \textbf{Importancia}          & Alta \\
    \hline
\end{longtable}

\begin{longtable}{ p{0.21\columnwidth} p{0.71\columnwidth} }
    \caption{CU-7 Introducir las credenciales de la red WiFi}\\
    \hline
    \textbf{CU-7}    & \textbf{Introducir las credenciales de la red WiFi}\\
    \hline
    \endfirsthead
    \hline
    \textbf{CU-7}    & \textbf{Introducir las credenciales de la red WiFi}\\
    \hline
    \endhead
        \hline
        \multicolumn{2}{c}{Sigue en la página siguiente.}
        \endfoot
        \hline
        \endlastfoot
    \textbf{Versión}              & 1.0    \\
    \textbf{Autor}                & Oscar Valverde Escobar \\
    \textbf{Requisitos asociados} & RF-9\\
    \textbf{Descripción}          & Operaciones realizadas para conectar el sistema a una red WiFi.\\
    \textbf{Precondición}         & 
        \begin{itemize}
            \item El sistema tiene que estar dentro del alcance de la red WiFi.
            \item El sistema tiene que estar encendido
            \item La pantalla táctil tiene que funcionar
            \item La app gráfica tiene que estar lanzada
        \end{itemize}\\
    \textbf{Acciones}             &
    \begin{enumerate}
        \def\labelenumi{\arabic{enumi}.}
        \tightlist
        \item Mediante la pantalla tactil se accede a la configuración del wifi.
        \item Se escribe el nombre de la red WiFi
        \item Se escribe la contraseña de la red WiFi
        \item Se pulsa en aceptar
        \item Se guardan los datos de la red wifi y pide reiniciar
        \item Se reinicia el sistema
    \end{enumerate}\\
    \textbf{Postcondición}        &
    \begin{itemize}
            \item Al encender el sistema indicará que está conectado a la red WiFi
    \end{itemize}\\
    \textbf{Excepciones}          & 
        \begin{enumerate}
        \item Red no accesible
        \end{enumerate} \\
    \textbf{Importancia}          & Alta \\
    \hline
\end{longtable}

\begin{longtable}{ p{0.21\columnwidth} p{0.71\columnwidth} }
    \caption{CU-8 Graficar los datos de lo sensores desde una aplicación móvil}\\
    \hline
    \textbf{CU-8}    & \textbf{Graficar los datos de lo sensores desde una aplicación móvil}\\
    \hline
    \endfirsthead
    \hline
    \textbf{CU-8}    & \textbf{Graficar los datos de lo sensores desde una aplicación móvil}\\
    \hline
    \endhead
        \hline
        \multicolumn{2}{c}{Sigue en la página siguiente.}
        \endfoot
        \hline
        \endlastfoot
    \textbf{Versión}              & 1.0    \\
    \textbf{Autor}                & Oscar Valverde Escobar \\
    \textbf{Requisitos asociados} & RF-10\\
    \textbf{Descripción}          & Operaciones realizadas para obtener un sensor de la base de datos, leer su valor real y almacenarlo de manera persistente.\\
    \textbf{Precondición}         & 
        \begin{itemize}
            \item El sistema tiene que estar conectado a la misma red WiFi que el dispositivo movil.
            \item El sistema tiene que estar encendido
            \item el Servidor API REST tiene que estar funcionando
        \end{itemize}\\
    \textbf{Acciones}             &
    \begin{enumerate}
        \def\labelenumi{\arabic{enumi}.}
        \tightlist
        \item La App móvil utiliza un \textit{endpoint} de la API REST para solicitarle los datos de los sensores a graficar.
        \item El sistema recupera todos los registros de datos solicitados.
        \item El sistema procesa los registros de datos por intervalos para calcular su valor medio en dichos intervalos.
        \item El sistema empaqueta en una lista de valores medios y otra lista de intervalos temporales.
        \item El servidor API REST convierte el objeto generado con las listas en un archivo JSON y se lo envía a la App movil
        \item La app móvil recibe el archivo con las listas de datos
        \item La app movil grafica las listas de datos recibidas.
    \end{enumerate}\\
    \textbf{Postcondición}        &
    \begin{itemize}
            \item El número y valor de las instancias en la base de datos debe ser el mismo tras el procesado de valores medios.
            \item Se tiene que haber mostrado una gráfica en el móvil con los datos recibidos
    \end{itemize}\\
    \textbf{Excepciones}          & 
        \begin{enumerate}
        \item Red no accesible
        \end{enumerate} \\
    \textbf{Importancia}          & media \\
    \hline
\end{longtable}