\capitulo{6}{Trabajos relacionados}

Actualmente en el mercado ya se ofrecen diversas alternativas a este producto, aunque todas tienen un enfoque más autónomo puramente dicho, lo cual en Green In House se ha eliminado esa parte de autosuficienca, para involucar lo máximo posible al usuario  en el desarrollo de su planata (en este caso el niño):

\textbf{Click and Grow}: Este es un jardín interior inteligente que permite cultivar plantas en casa. Viene con cápsulas de semillas que se insertan en el jardín inteligente para un crecimiento sin problemas. Los sensores y el sistema automatizado de riego aseguran que las plantas reciban la cantidad adecuada de agua, luz y nutrientes. Los precios varían desde alrededor de 100€ hasta 200€ dependiendo del tamaño del jardín inteligente.

\textbf{AeroGarden}: Este es otro sistema de jardín interior inteligente que permite cultivar plantas en interiores durante todo el año. También utiliza semillas en cápsulas y tiene un sistema de iluminación LED ajustable y un panel de control para recordar cuándo agregar agua y nutrientes. Los precios oscilan entre 100€ y 300€ dependiendo del modelo.

\textbf{Planty Square}: Este es un jardín modular inteligente que permite a los usuarios cultivar varias plantas a la vez. Cada módulo puede cultivar una planta y los módulos se pueden conectar entre sí para formar un jardín más grande. Cuenta con una aplicación que envía notificaciones para regar las plantas. El precio ronda los 100€ .

\textbf{SproutsIO}: Este otro sistema es un jardín interior de alta tecnología que permite cultivar plantas sin tierra. Utiliza la hidroponía y la aeroponía para cultivar plantas y tiene una aplicación que permite a los usuarios monitorear y controlar su jardín desde su teléfono. El precio es más elevado, alrededor de 800€.

Es importante aclarar que estos precios pueden variar por región debido a factores como los impuestos, los costos de envío y la tasa de cambio. Todos estos productos tienen en común la idea de utilizar la tecnología para facilitar el cultivo de plantas en interiores, pero cada uno tiene sus propias características únicas, al igual que Green In House. Para determinar un precio de venta para Green In House, habría que tener en cuenta el costo de producción, las características del producto, el precio de los competidores y la disposición a pagar del público, lo cual se detallará más adelante.





Comparativa de productos en el mercado con mi solucion

Este apartado sería parecido a un estado del arte de una tesis o tesina. En un trabajo final grado no parece obligada su presencia, aunque se puede dejar a juicio del tutor el incluir un pequeño resumen comentado de los trabajos y proyectos ya realizados en el campo del proyecto en curso. 
